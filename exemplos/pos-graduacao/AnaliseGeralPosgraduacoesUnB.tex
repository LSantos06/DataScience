\documentclass[]{article}
\usepackage{lmodern}
\usepackage{amssymb,amsmath}
\usepackage{ifxetex,ifluatex}
\usepackage{fixltx2e} % provides \textsubscript
\ifnum 0\ifxetex 1\fi\ifluatex 1\fi=0 % if pdftex
  \usepackage[T1]{fontenc}
  \usepackage[utf8]{inputenc}
\else % if luatex or xelatex
  \ifxetex
    \usepackage{mathspec}
  \else
    \usepackage{fontspec}
  \fi
  \defaultfontfeatures{Ligatures=TeX,Scale=MatchLowercase}
\fi
% use upquote if available, for straight quotes in verbatim environments
\IfFileExists{upquote.sty}{\usepackage{upquote}}{}
% use microtype if available
\IfFileExists{microtype.sty}{%
\usepackage{microtype}
\UseMicrotypeSet[protrusion]{basicmath} % disable protrusion for tt fonts
}{}
\usepackage[margin=1in]{geometry}
\usepackage{hyperref}
\hypersetup{unicode=true,
            pdftitle={Ciência de Dados para Todos (Data Science For All) - 2018.1 - Análise da Produção Científica e Acadêmica da Universidade de Brasília - Modelo de Relatório Final da Disciplina - Departamento de Ciência da Computação da UnB},
            pdfauthor={Jorge H. C. Fernandes, Ricardo B. Sampaio, João Ribas de Moura e Jerônimo A. Filho},
            pdfborder={0 0 0},
            breaklinks=true}
\urlstyle{same}  % don't use monospace font for urls
\usepackage{color}
\usepackage{fancyvrb}
\newcommand{\VerbBar}{|}
\newcommand{\VERB}{\Verb[commandchars=\\\{\}]}
\DefineVerbatimEnvironment{Highlighting}{Verbatim}{commandchars=\\\{\}}
% Add ',fontsize=\small' for more characters per line
\usepackage{framed}
\definecolor{shadecolor}{RGB}{248,248,248}
\newenvironment{Shaded}{\begin{snugshade}}{\end{snugshade}}
\newcommand{\KeywordTok}[1]{\textcolor[rgb]{0.13,0.29,0.53}{\textbf{#1}}}
\newcommand{\DataTypeTok}[1]{\textcolor[rgb]{0.13,0.29,0.53}{#1}}
\newcommand{\DecValTok}[1]{\textcolor[rgb]{0.00,0.00,0.81}{#1}}
\newcommand{\BaseNTok}[1]{\textcolor[rgb]{0.00,0.00,0.81}{#1}}
\newcommand{\FloatTok}[1]{\textcolor[rgb]{0.00,0.00,0.81}{#1}}
\newcommand{\ConstantTok}[1]{\textcolor[rgb]{0.00,0.00,0.00}{#1}}
\newcommand{\CharTok}[1]{\textcolor[rgb]{0.31,0.60,0.02}{#1}}
\newcommand{\SpecialCharTok}[1]{\textcolor[rgb]{0.00,0.00,0.00}{#1}}
\newcommand{\StringTok}[1]{\textcolor[rgb]{0.31,0.60,0.02}{#1}}
\newcommand{\VerbatimStringTok}[1]{\textcolor[rgb]{0.31,0.60,0.02}{#1}}
\newcommand{\SpecialStringTok}[1]{\textcolor[rgb]{0.31,0.60,0.02}{#1}}
\newcommand{\ImportTok}[1]{#1}
\newcommand{\CommentTok}[1]{\textcolor[rgb]{0.56,0.35,0.01}{\textit{#1}}}
\newcommand{\DocumentationTok}[1]{\textcolor[rgb]{0.56,0.35,0.01}{\textbf{\textit{#1}}}}
\newcommand{\AnnotationTok}[1]{\textcolor[rgb]{0.56,0.35,0.01}{\textbf{\textit{#1}}}}
\newcommand{\CommentVarTok}[1]{\textcolor[rgb]{0.56,0.35,0.01}{\textbf{\textit{#1}}}}
\newcommand{\OtherTok}[1]{\textcolor[rgb]{0.56,0.35,0.01}{#1}}
\newcommand{\FunctionTok}[1]{\textcolor[rgb]{0.00,0.00,0.00}{#1}}
\newcommand{\VariableTok}[1]{\textcolor[rgb]{0.00,0.00,0.00}{#1}}
\newcommand{\ControlFlowTok}[1]{\textcolor[rgb]{0.13,0.29,0.53}{\textbf{#1}}}
\newcommand{\OperatorTok}[1]{\textcolor[rgb]{0.81,0.36,0.00}{\textbf{#1}}}
\newcommand{\BuiltInTok}[1]{#1}
\newcommand{\ExtensionTok}[1]{#1}
\newcommand{\PreprocessorTok}[1]{\textcolor[rgb]{0.56,0.35,0.01}{\textit{#1}}}
\newcommand{\AttributeTok}[1]{\textcolor[rgb]{0.77,0.63,0.00}{#1}}
\newcommand{\RegionMarkerTok}[1]{#1}
\newcommand{\InformationTok}[1]{\textcolor[rgb]{0.56,0.35,0.01}{\textbf{\textit{#1}}}}
\newcommand{\WarningTok}[1]{\textcolor[rgb]{0.56,0.35,0.01}{\textbf{\textit{#1}}}}
\newcommand{\AlertTok}[1]{\textcolor[rgb]{0.94,0.16,0.16}{#1}}
\newcommand{\ErrorTok}[1]{\textcolor[rgb]{0.64,0.00,0.00}{\textbf{#1}}}
\newcommand{\NormalTok}[1]{#1}
\usepackage{longtable,booktabs}
\usepackage{graphicx,grffile}
\makeatletter
\def\maxwidth{\ifdim\Gin@nat@width>\linewidth\linewidth\else\Gin@nat@width\fi}
\def\maxheight{\ifdim\Gin@nat@height>\textheight\textheight\else\Gin@nat@height\fi}
\makeatother
% Scale images if necessary, so that they will not overflow the page
% margins by default, and it is still possible to overwrite the defaults
% using explicit options in \includegraphics[width, height, ...]{}
\setkeys{Gin}{width=\maxwidth,height=\maxheight,keepaspectratio}
\IfFileExists{parskip.sty}{%
\usepackage{parskip}
}{% else
\setlength{\parindent}{0pt}
\setlength{\parskip}{6pt plus 2pt minus 1pt}
}
\setlength{\emergencystretch}{3em}  % prevent overfull lines
\providecommand{\tightlist}{%
  \setlength{\itemsep}{0pt}\setlength{\parskip}{0pt}}
\setcounter{secnumdepth}{0}
% Redefines (sub)paragraphs to behave more like sections
\ifx\paragraph\undefined\else
\let\oldparagraph\paragraph
\renewcommand{\paragraph}[1]{\oldparagraph{#1}\mbox{}}
\fi
\ifx\subparagraph\undefined\else
\let\oldsubparagraph\subparagraph
\renewcommand{\subparagraph}[1]{\oldsubparagraph{#1}\mbox{}}
\fi

%%% Use protect on footnotes to avoid problems with footnotes in titles
\let\rmarkdownfootnote\footnote%
\def\footnote{\protect\rmarkdownfootnote}

%%% Change title format to be more compact
\usepackage{titling}

% Create subtitle command for use in maketitle
\newcommand{\subtitle}[1]{
  \posttitle{
    \begin{center}\large#1\end{center}
    }
}

\setlength{\droptitle}{-2em}

  \title{Ciência de Dados para Todos (Data Science For All) - 2018.1 - Análise da
Produção Científica e Acadêmica da Universidade de Brasília - Modelo de
Relatório Final da Disciplina - Departamento de Ciência da Computação da
UnB}
    \pretitle{\vspace{\droptitle}\centering\huge}
  \posttitle{\par}
    \author{Jorge H. C. Fernandes, Ricardo B. Sampaio, João Ribas de Moura e
Jerônimo A. Filho}
    \preauthor{\centering\large\emph}
  \postauthor{\par}
      \predate{\centering\large\emph}
  \postdate{\par}
    \date{11/06/2018}


\begin{document}
\maketitle

\section{Introdução}\label{introducao}

Este documento apresenta um modelo básico para a construção do relatório
final da disciplina Tópicos Avançados em Computadores - TurmaD - 2018.2,
do Departamento de Ciência da Computação da Universidade de Brasília,
que trata da análise da produção científica e acadêmica na Universidade
de Brasília, em uma ou mais áreas de conhecimento.

A metodologia para desenvolvimento do relatório é baseada no modelo de
mineração de dados denominado CRISP-DM (Chapman et al., 2000, Mariscal
et al., 2010).

Este documento deve ser referenciado do modo como aparece na seção de
referências ao final do texto, abaixo reproduzida.

Fernandes, Jorge H C, Ricardo Barros Sampaio, João Ribas de Moura e
Jerônimo AVelar Filho. ``Ciência de Dados para Todos (Data Science For
All) - 2018.1 - Análise da Produção Científica e Acadêmica da
Universidade de Brasília - Modelo de Relatório Final da Disciplina -
Departamento de Ciência da Computação da UnB''. Disciplina 116297 -
Tópicos Avançados em Computadores, turma D, do semestre 2018.1, do
Departamento de Ciência da Computação do Instituto de Ciências Exatas da
Universidade de Brasília, 13 de junho de 2018.

\section{CRISP-DM (Corresponderia à seção de
Metodologia)}\label{crisp-dm-corresponderia-a-secao-de-metodologia}

Para desenvolvimento do trabalho devem ser seguidos, da forma mais
simplificada e coerente possível, as fases e atividades genéricas do
ciclo de vida de um projeto executado em aderência ao CRISP-DM, conforme
ilustra de forma geral a figura 1. Em outras palavras, a produção do
relatorio deve seguir a metodologia CRISP-DM.

Perceba que a Figura 1 sugere haver grande flexibilidade na execução das
fases, de modo que se pode retornar a fases anteriores em muitos pontos.

``A widely used methodology for data mining is the CRoss-Industry
Standard Process for Data Mining (CRISP-DM) which mas initiated in 1996
(\ldots{}) with the intent of providing a process that is
\textbf{reliable and repeatable} by people with little data-mining
background, with a framework within which experience can be recorded, to
support the replication of projects, to support planning and management,
as well as to demonstrate data mining as a mature discipline
(\ldots{})'' {[}Sullivan, Rob. Introduction to Data Mining for the Life
Sciences. Springer Science \& Business Media. 2012{]}

O seu trabalho deve conter uma seção metodologia, onde você faz uma
breve descrição da metodologia que seu grupo adotou para realização do
trabalho, que pode ser baseada no texto dessa seção, desde que citado
adequadamente.

\subsection{Delimitações iniciais}\label{delimitacoes-iniciais}

Em aderência à estrutura do CRISP-DM, algumas delimitações de contexto
para o trabalho são apresentadas a seguir:

\subsubsection{Domínio de Aplicação do
projeto}\label{dominio-de-aplicacao-do-projeto}

O domínio de aplicação do projeto é o da produção científica e acadêmica
brasileira, mais específicamente a produção científica ou produção
acadêmica de um subgrupo de pesquisadores vinculados à Universidade de
Brasília. O domínio de aplicação do projeto deve ser declarado na
introdução ao relatório.

\subsubsection{Tipo de Problema
abordado}\label{tipo-de-problema-abordado}

O tipo de problema abordado é o da produção de análises descritivas,
quantitativas e de modelagem computacional ou estatística, que permitam
caracterizar como e porque ocorre a produção científica e acadêmica de
um grupo de pesquisadores. Essa caracterização visa subsidiar a tomada
de decisão por membros do Sistema Nacional de Pós-Graduação. O tipo de
problema abordado no projeto deve ser declarado na introdução ao
relatório.

\subsubsection{Conjunto de Ferramentas e
Técnicas}\label{conjunto-de-ferramentas-e-tecnicas}

O conjunto de requisitos relativos a ferramentas e técnicas a serem
empregadas na execução do trabalho é:

\begin{itemize}
\tightlist
\item
  O relatório deve ser entregue no formato R Markdown, apto à geração de
  saída \LaTeX e PDF, composto por comandos em R entremeados por
  descrições textuais que auxiliem na interpretação dos resultados, bem
  como na compreensão do domínio de conhecimento sob análise.
\item
  As análises descritivas devem empregar de forma criativa as funções
  das bibliotecas de ciência de dados em R propostas por Wickham e
  Grolemund (2016).
\item
  As análises quantitativas devem lançar mão de recursos gráficos
  variados, que complementarão análises descritivas com \emph{insights}
  sobre de que forma os processos de produção científica e acadêmica
  contribuem para os resultados apresentados. Por exemplo, os dados
  analisados possibilitam justificar o eventual crescimento ou
  decréscimo de índices de produção observados?
\item
  A modelagem computacional ou estatística avançada dos dados deve usar
  uma das quatro técnicas prescritas:

  \begin{itemize}
  \tightlist
  \item
    Aprendizado de Máquina (Datacamp, 2018; Kuhn et al., 2018; Bruce e
    Bruce, 2017);
  \item
    Aprendizado Estatístico;
  \item
    Mineração de Texto ou;
  \item
    Análise de Redes (Kolaczyk e Csárdi, 2014; Lusher et al., 2013; de
    Nooy et al., 2005).
  \end{itemize}
\end{itemize}

O conjunto de requisitos relativos a ferramentas e técnicas a serem
empregadas na execução do trabalhodo projeto deve ser declarado na parte
de metodologia do relatório.

\subsection{Modelo de Referência
CRISP-DM}\label{modelo-de-referencia-crisp-dm}

Miner (2012), aprofunda: ``(\ldots{}) In CRISP-DM, the complete life
cycle of a data mining project is represented with \textbf{six phases}:
business understanding (determining the purpose of the study), data
understanding (data exploration and understanding), data preparation,
modeling, evaluation, and deployment.(\ldots{}). {[}Miner, Gary.
Practical Text Mining and Statistical Analysis for Non-structure Text
Data Applications. Academic Press, 2012.{]}

\subsubsection{Por que usar o CRISP-DM?}\label{por-que-usar-o-crisp-dm}

Imagine uma analogia entre um projeto de datamining e a preparação de
uma receita de bolo para ser usada em uma fábrica. Para iniciar a
produção, com base numa receita de comprovada eficácia (metodológica e
científica), você tem que minerar os ingredientes (dados) em um grande
supermercado (\emph{dataset}). Com os ingredientes você precisa aplicar
um método (a forma de misturá-los), colocar os ingredientes numa
determinada ordem, mexer por um certo tempo, aquecer por tantos minutos
até o bolo ficar pronto e ser aprovado em um ou mais testes de
degustação.

Tendo por objetivo fazer com que essa receita (script de mineração de
dados) possa ser executada com sucesso diversas vezes, numa fábrica,
será que outro cozinheiro (cientista) que reproduzisse a receita
(método) chegaria ao mesmo resultado? Se a metodologia (receita) já foi
bastante testada, então é bem provável que o resultado será o mesmo e
seu produto (receita de bolo) será aceito para a produção
(\emph{deployment}) de análises para consumo futuro, com base em
fundamentos científicos.

\subsubsection{Organização hierárquica de atividades em
fases}\label{organizacao-hierarquica-de-atividades-em-fases}

Dentro de cada fase no CRISP-DM existe uma estrutura hierárquica de
atividades genéricas para serem realizadas. Cada uma dessas atividades
\textbf{genéricas} pode determinar a execução de atividades
\textbf{específicas}.

Voltando ao exemplo do bolo, a atividade '' 1. Entendimento do Bolo''
poderia conter uma atividade genérica chamada ``1.1. Determinar para que
o bolo servirá (simples café da manhã? bolo de aniversário? bolo de
casamento?)``. Dentro dessa atividade genérica poderia haver atividades
específicas como ``1.1.1.Entrevistar o contratante para obter detalhes
de onde o bolo será usado?``; ``1.1.2. Conversar com os convidados sob
alguma necessidade especial (sem lactose? sem glútem?)``, etc.

\subsubsection{Seis Fases do CRISP-DM}\label{seis-fases-do-crisp-dm}

Com base no apresentado, segue uma descrição um pouco mais detalhada das
seis fases de um projeto no CRISP-DM, interpretadas no contexto do
relatório que você e seu grupo deverão produzir.

Todas as fases deverão ser adequadamente relatadas no relatório, em
seções que aparecem após a seçãoda metodologia

\begin{enumerate}
\def\labelenumi{\arabic{enumi}.}
\tightlist
\item
  O propósito da fase de \textbf{Entendimento do Negócio} é o
  desenvolvimento dos objetivos e declaração das necessidades do projeto
  sob a perspectiva do negócio, para transformar isso tudo em definição
  de um problema de data mining.
\end{enumerate}

As atividades genéricas dentro dessa fase envolvem:

\begin{itemize}
\item
  Identificar o que a organização realmente necessita alcançar. No caso
  específico desta disciplina, a necessidade do Sistema Nacional de
  Pós-Graduação do Brasil de produzir análises de alta qualidade de suas
  pós-graduações, com baixo custo. Como produzir um projeto de mineração
  de dados se você não sabe o que necessita encontrar ou resolver? Se
  você não entender os objetivos da organização pode levar ao erro de
  procurar as respostas certas para as perguntas erradas.
\item
  Avaliação das Circunstâncias. Envolve identificar quais recursos ou
  dificuldades podem influenciar os objetivos da mineração ou do projeto
  em si. No caso específico desta disciplina, isso envolve refletir,
  entre vários outros aspectos, sobre as limitações de tempo do projeto,
  que precisa ser realizado dentro de um semestre letivo, de modo que
  considerável parte das atividades já foram pré-organizadas pelos
  docentes responsáveis pela disciplina.
\item
  O projeto de mineração é o grande objetivo desta etapa e o relatório
  precisa conter uma seção sobre Metodologia, apresentando em detalhes o
  que se pretende fazer adiante.
\end{itemize}

\begin{enumerate}
\def\labelenumi{\arabic{enumi}.}
\setcounter{enumi}{1}
\tightlist
\item
  A fase de \textbf{Entendimento dos Dados} inicia determinando quais
  são os dados realmente disponíveis na organização, se existe permissão
  para utilizá-los, se existem dados confidenciais ou cobertos pelo
  sigilo. Por exemplo, um \emph{dataset} das declarações de imposto de
  renda da Receita Federal certamente seria protegido pelo sigilo
  fiscal. Dados de pacientes de hospitais podem conter restrições.
\end{enumerate}

Também é necessário acessar os dados para compreendê-los melhor para ter
o \emph{insight} de como será feita a modelagem mais tarde.

Na fase de entendimento dos dados pode-se trabalhar com quatro
atividades genéricas:

\begin{itemize}
\item
  Coleta inicial dos dados. Essa atividade envolve a análise das
  permissões de acesso e outras questões envolvendo sigilo e outros
  proprietários dos dados (terceiros). Por exemplo, eu poderia estar
  acessando uma base de dados que foi obtida de outro órgão por
  convênio, mas nesse convênio (contrato) não foi dada permissão para
  qualquer outro tipo de acesso ou exploração dos dados. Neste projeto,
  a coleta inicial foi feita pelos autores deste relatório. O relatório
  final deve conter indicações de como foi realizada a coleta inicial
  dos dados.
\item
  Descrição dos dados. A descrição dos dados verifica se os dados sendo
  acessados terão potencial para responder às questões de \emph{data
  mining}. Além disso, deve-se avaliar qual o volume de dados, a
  estrutura dos dados (tipos), codificações usadas, etc. Neste projeto,
  a descrição dos dados é responsabilidade parcial dos alunos, tendo em
  vista que este modelo já oferece uma descrição inicial. O relatório
  final deve conter descrições significativas e aprofundadas dos dados.
\item
  Análise exploratória dos dados. A análise exploratória dos dados
  possibilita um entendimento mais profundo da relação estatística
  existente entre os dados dos \emph{datasets} para um melhor
  entendimento da qualidade daqueles dados para o objetivo do projeto.
  Neste projeto, a análise exploratória dos dados é responsabilidade
  parcial dos alunos, tendo em vista que este relatório apresenta uma
  análise exploratória preliminar. O relatório final deve conter
  análises exploratórias dos dados que sejam significativas e
  aprofundadas.
\item
  Verificação da qualidade dos dados. A verificação da qualidade dos
  dados envolve responder se os dados disponíveis estão realmente
  completos. As informações disponíveis são suficientes para o trabalho
  proposto? Neste projeto, a verificação da qualidade dos dados é
  responsabilidade dos alunos.
\end{itemize}

\begin{enumerate}
\def\labelenumi{\arabic{enumi}.}
\setcounter{enumi}{2}
\tightlist
\item
  Na fase de \textbf{Preparação dos Dados} os \emph{datasets} que serão
  utilizados em todo o trabalho são construídos a partir dos dados
  brutos. Aqui os dados são ``filtrados'' retirando-se partes que não
  interessam e selecionando-se os ``campos'' necessários para o trabalho
  de mineração.
\end{enumerate}

São 5 as atividades genéricas nesta fase de preparação dos dados:

\begin{itemize}
\item
  Seleção dos dados. Envolve identificar quais dados, da nossa
  ``montanha de dados'', serão realmente utilizados. Quais variáveis dos
  dados brutos serão convertidas para o \emph{dataset}? Não é raro
  cometer o erro de selecionar dados para um modelo preditivo com base
  em uma falsa ideia de que aqueles dados contém a resposta para o
  modelo que se quer construir. Surge o cuidado de se separar o sinal do
  ruído (Silver, Nate. The Signal and the Noise: Why so many predictions
  fail --- but some don't. USA: The Penguin Press HC, 2012.).
\item
  Limpeza dos dados.
\item
  Construção dos dados. Envolve a criação de novas variáveis a partir de
  outras presentes nos \emph{datasets}.
\item
  Integração dos dados. Envolve a união (merge) de diferentes tabelas
  para criar um único \emph{dataset} para ser utilizado no R, por
  exemplo.
\item
  Formatação dos dados. Envolve a realização de pequenas alterações na
  estrutura dos dados, como a ordem das variáveis, para permitir a
  execução de determinado método de data mining.
\end{itemize}

\begin{enumerate}
\def\labelenumi{\arabic{enumi}.}
\setcounter{enumi}{3}
\tightlist
\item
  A fase de \textbf{Modelagem} no CRISP-DM envolve a construção e
  avaliação do modelo, podendo ser realizada em quatro atividades
  genéricas:
\end{enumerate}

\begin{itemize}
\item
  Seleção das técnicas de modelagem.
\item
  Realização de testes de modelagem, onde diferentes modelos são
  previamente testados e avaliados. Pode-se dividir o \emph{dataset}
  criado na etapa anterior para se ter uma base de treino na construção
  de modelos, e outra pequena parte para validar e avaliar a eficiência
  de cada modelo criado até se chegar ao mais ``eficiente''.
\item
  Construção do modelo definitivo, com base na melhor experiência do
  passo anterior.
\item
  Avaliação do modelo.
\end{itemize}

\begin{enumerate}
\def\labelenumi{\arabic{enumi}.}
\setcounter{enumi}{4}
\tightlist
\item
  Na fase de \textbf{Avaliação} do CRISP-DM os resultados não são apenas
  avaliados, mas se verifica se existem questões relacionadas à
  organização que não foram suficientemente abordadas. Deve-se refletir
  se o uso arepetido do modelo criado pode trazer algum ``efeito
  colateral'' para a organização.
\end{enumerate}

Nesta fase, pode-se trabalhar com 3 atividades genéricas:

\begin{itemize}
\item
  Avaliação dos resultados
\item
  Revisão do processo, por meio da qual verifica-se se o modelo foi
  construído adequadamente. As variáveis (passadas) para construir o
  modelo estarão disponíveis no futuro?
\item
  Determinação dos etapas seguintes. Pode ser necessário decidir-se por
  finalizar o projeto, passar à etapa de desenvolvimento, ou rever
  algumas fases anteriores para a melhoria do projeto.
\end{itemize}

\begin{enumerate}
\def\labelenumi{\arabic{enumi}.}
\setcounter{enumi}{5}
\tightlist
\item
  Na fase de \textbf{Implantação} (\emph{deployment}) se realiza o
  planejamento de implantação dos produtos desenvolvidos (scripts, no
  caso do executado nesta disciplina) para o ambiente operacional, para
  seu uso repetitivo, envolvendo atividades de monitoramento e
  manutenção do sistema (script) desenvolvido. A fase de implantação
  concluir com a produção e apresentação do relatório final com os
  resultados do projeto.
\end{enumerate}

São atividades genéricas na fase de \textbf{implantação}:

\begin{itemize}
\tightlist
\item
  Planejamento da transição dos produtos;
\item
  Planejamento do monitoramento dos produtos em utilização no ambiente
  operacional;
\item
  Planejamento de manuteção a ser eventualmente efetuada no produto
  (scripts);
\item
  Produção do relatório final;
\item
  Apresentação do relatório final;
\item
  Revisão sobre a execução do projeto, com registro de lições aprendidas
  etc.
\end{itemize}

No contexto do projeto realizado no âmbito desta disciplina, a
responsabilidade por execução de todas essas atividades é dos alunos,
com exceção da apresentação do relatório final, que não será realizada.

\section{CRISP-DM Fase 1 - Entendimento do
Negócio}\label{crisp-dm-fase-1---entendimento-do-negocio}

\subsection{O que é o Sistema Nacional de Pós-Graduação?
(Contextualização)}\label{o-que-e-o-sistema-nacional-de-pos-graduacao-contextualizacao}

A produção do conhecimento científico, no Brasil, é predominantemente
efetuada por meio do Sistema Nacional de Pós-Graduação - SNPG, e mais
fortemente relacionada com a formação de doutores nesse sistema (Pátaro
e Mezzomo, 2013), por meio de cursos de pós-graduação \emph{strictu
sensu}.

Fernandes e Sampaio (2017) já indicaram que a ciência é reconhecidamente
um elemento essencial para o desenvolvimento social e econômico de
qualquer nação. Assim sendo, faz-se mister aprimorar o SNPG como forma
de promoção desse crescimento, visando maximizar o retorno decorrente do
emprego dos recursos nele aplicados. A promoção do crescimento do SNPG
se dá predominantemente por meio de avaliações regulares de seus
programas de pós-graduação, sob responsabilidade da CAPES, que realiza a
cada quatro anos um complexo (Leite, 2018, p.~13) e custoso processo de
coleta de dados, análise e deliberação sobre as pós-graduações
\emph{strictu sensu}, em coerência com o estabelecido no Plano Nacional
de Pós-Graduação (PNPG) 2012-2020 (CAPES, 2010) e nos diversos
documentos que definem os critérios de organização da pós-graduação em
cada área do conhecimento (CAPES, 2018). Leite (2018) faz uma
apresentação geral de como se organizam e são avaliadas as
pós-graduações no Brasil.

O Plano Nacional de Pós-Graduação (PNPG), por outro lado, define
diretrizes estratégicas para desenvolvimento da pós-graduação
brasileira, que deve abordar prioritariamente grandes temas de interesse
nacional, tais como a redução das assimetrias de desenvolvimento entre
as regiões do Brasil, a formação de professores para a educação básica,
a formação de recursos humanos para as empresas, a resposta aos grandes
desafios brasileiros sobre Água, Energia, Transporte, Controle de
Fronteiras, Agronegócio, Amazônia, Amazônia Azul (Mar), Saúde, Defesa,
Programa Espacial, além de Justiça, Segurança Pública, Criminologia e
Desequilíbrio Regional. O PNPG também traça as diretrizes para
financiamento da pós-graduação e sua internacionalização, apresentando
conclusões e recomendações.

As avaliações do SNPG, ao atribuirem mensurações de desempenho às
diversas pós-graduações que dele fazem parte, geram incentivos e
penalidades aos programas, tendo em vista a limitada disponibilidade de
recursos para investimento em bolsas, taxas de bancada etc. Embora o
sistema seja altamente sofisticado ele é também altamente criticado
(Azevedo et al., 2016), sobretudo porque há percalços na busca por um
equilíbrio entre as diferentes concepções de finalidade da ciência. Se
de um lado a promoção do conhecimento gerado predominantemente nas ditas
ciências \emph{hard} constribui para criar fluxos econômicos mais
intensos, isso não significa que essa promoção possa ocorrer em
detrimento da menor promoção na geração de conhecimento sobre problemas
sociais, predominantemente gerado nas ditas ciências \emph{soft},
especialmente das áreas de humanidades, sob pena de ampliação de
desigualdades (Azevedo et al., 2016).

Não há solução simples, mas postula-se, nesta disciplina, que uma maior
agilidade na avaliação e a utilização de critérios mais objetivos,
poderá facilitar a melhoria do sistema.

\subsubsection{Os Colégios, Grandes Áreas e Áreas da Pós-Graduação
Brasileira}\label{os-colegios-grandes-areas-e-areas-da-pos-graduacao-brasileira}

A partir de 2018, as diversas áreas da pós-graduação brasileira foram
organizadas na forma de colégios, grandes áreas e áreas, conforme
apresentam as tabelas a seguir.

\paragraph{Colégio de Ciências da
vida}\label{colegio-de-ciencias-da-vida}

\begin{longtable}[]{@{}lll@{}}
\toprule
CIÊNCIAS AGRÁRIAS & CIÊNCIAS BIOLÓGICAS & CIÊNCIAS DA
SAÚDE\tabularnewline
\midrule
\endhead
Ciência de Alimentos & Biodiversidade & Educação Física\tabularnewline
Ciências Agrárias I & Ciências Biológicas I & Enfermagem\tabularnewline
Medicina Veterinária & Ciências Biológicas II & Farmácia\tabularnewline
Zootecnia / Recursos Pesqueiros & Ciências Biológicas III & Medicina
I\tabularnewline
- & - & Medicina II\tabularnewline
- & - & Medicina III\tabularnewline
- & - & Nutrição\tabularnewline
- & - & Odontologia\tabularnewline
- & - & Saúde Coletiva\tabularnewline
\bottomrule
\end{longtable}

\paragraph{Colégio de Ciências Exatas, Tecnológicas e
Multidisciplinar}\label{colegio-de-ciencias-exatas-tecnologicas-e-multidisciplinar}

\begin{longtable}[]{@{}lll@{}}
\toprule
CIÊNCIAS EXATAS E DA TERRA & ENGENHARIAS &
MULTIDISCIPLINAR\tabularnewline
\midrule
\endhead
Astronomia / Física & Engenharias I & Biotecnologia\tabularnewline
Ciência da Computação & Engenharias II & Ciências
Ambientais\tabularnewline
Geociências & Engenharias III & Ensino\tabularnewline
Matemática / Probabilidade e Estatística & Engenharias IV &
Interdisciplinar\tabularnewline
Química & - & Materiais\tabularnewline
\bottomrule
\end{longtable}

\paragraph{Colégio de Humanidades}\label{colegio-de-humanidades}

\begin{longtable}[]{@{}lll@{}}
\toprule
CIÊNCIAS HUMANAS & CIÊNCIAS SOCIAIS APLICADAS & LINGUÍSTICA, LETRAS E
ARTES\tabularnewline
\midrule
\endhead
Antropol/Arqueol & Admin.Púb./Empr.,C.Contáb. e Tur. &
Artes\tabularnewline
Ciência Pol. e Rel. Int. & Arquit., Urban. e Design & Linguística e
Literatura\tabularnewline
Ciências da Religião e Teol. & Comunicação e Informação &
-\tabularnewline
Educação & Direito & -\tabularnewline
Filosofia & Economia & -\tabularnewline
Geografia & Planej. Urbano e Reg. / Demografia & -\tabularnewline
História & Serviço Social & -\tabularnewline
Psicologia & - & -\tabularnewline
Sociologia & - & -\tabularnewline
\bottomrule
\end{longtable}

Cada um desses colégios, grandes áreas e áreas de conhecimento possuem
dinâmicas próprias, e, portanto, não há um modelo universal que se
aplique a todas. Existem aspectos comuns, mas também grandes
peculiaridades, descritas parcialmente nos correspondentes documentos de
área disponíveis em CAPES (2018).

\subsection{A UnB dentro do Sistema Nacional de Pós-Graduação
(Contextualização)}\label{a-unb-dentro-do-sistema-nacional-de-pos-graduacao-contextualizacao}

\subsubsection{O que é a UnB?}\label{o-que-e-a-unb}

Descrição da Universidade de Brasília, com foco na sua produção
científica e acadêmica.

\subsubsection{Descrição das pós-graduações da
UnB}\label{descricao-das-pos-graduacoes-da-unb}

Texto a desenvolver.

\subsubsection{Outros aspectos que caracterizam a produção científica e
acadêmica da
UnB}\label{outros-aspectos-que-caracterizam-a-producao-cientifica-e-academica-da-unb}

Texto a desenvolver.

\subsection{O que a Organização precisa realmente
alcançar?}\label{o-que-a-organizacao-precisa-realmente-alcancar}

Vários stakeholders estão envolvidos no projeto em curso, e poderíamos
considerar cada um deles como distintas organizações que possuem
interesses distintos e complementares. Elas são: * A Disciplina Ciência
de Dados para Todos 2018.1, que quer comprovar que seus alunos dominam
ferramentas e técnicas de ciência de dados, para fins de avaliação de
rendimento da disciplina. * A UnB, representada pelos decanatos de
pós-graduação (DPG) e de pesquisa e inovação (DPI), que querem dispor de
instrumentos para realização de avaliações contínuas de suas
pós-graduações. * O SNPG, que assim com o DPG e DPI, também pode se
beneficiar do uso de instrumentos para realização de avaliações
contínuas de suas pós-graduações. * Os interessados em melhor conhecer o
que é produzido pelo Sistema Nacional de Pós-graduação, como empresas
privadas, que querem desfrutar dos benefícios gerados pela ciência
brasileira.

A fim de dar maior fidelidade e homogeneidade ao exercício realizado na
disciplina, focaremos em atendimento aos interesses comuns das
organizações DPI, DPG e CAPES, que desejam dispor de instrumentos ágeis
para avaliação contínua da pós-graduação brasileira.

Com base no exposto, o objetivo do trabalho final a ser alcançado pelos
produtos d emineração de dados desenvolvido pelos alunos da disciplina
Ciência de Dados para Todos é produzir, tomando por base inicial os
dados fornecidos pelos professores responsáveis pela disciplina,
ferramentas para análise e avaliação contínuas e de baixo custo, do
desempenho de um conjunto de pós-graduações que estão vinculadas a uma
mesma subárea ou grupo de conhecimento. Cada área de pós-graduação
apresenta suas características peculiares, assim como cada um dos
programas vinculados a essas áreas. Como já informado, características
peculiares de cada programa podem ser obtidas a partir de visita ao
sítio da CAPES (2018).

\subsection{Avaliação das
Circunstâncias}\label{avaliacao-das-circunstancias}

Este documento serve como base para a realização dos trabalhos dos
alunos. apresenta limitações no tocante à quantidade pequena de dados
que serão empregados para análises e avaliações, tendo em vista sua
finalidade maior que é a didática, de permitir aos alunos demonstrarem a
capacidade de aplicação das técnicas e ferramentas apreendidas durante o
semestre.

\subsubsection{Avaliação preliminar das pós-graduações na
UnB}\label{avaliacao-preliminar-das-pos-graduacoes-na-unb}

Texto a desenvolver.

\subsubsection{Avaliação preliminar da produção científica e acadêmica
da
UnB}\label{avaliacao-preliminar-da-producao-cientifica-e-academica-da-unb}

Texto a desenvolver.

\section{CRISP-DM Fase 2 - Entendimento dos
Dados}\label{crisp-dm-fase-2---entendimento-dos-dados}

Doravante, a fim de facilitar aos alunos seguirem a metodologia
CRISP-DM, os nomes das seções e subseções de texto serão prefixadas com
o número e nome da fase e atividade genérica do CRISP-DM. Fica facultado
aos grupos seguir ou não a sequência prevista, tendo em vista que se
pode retornar às fases anteriores, bem como podem haver atividades que
não foram adequadas às características do problema específico sob
análise.

\subsection{CRISP-DM Fase.Atividade 2.1 - Coleta inicial dos
dados}\label{crisp-dm-fase.atividade-2.1---coleta-inicial-dos-dados}

Todos os arquivos com dados iniciais a seguir apresentados foram
fornecidos pelos professores responsáveis pela disciplina. Os dados
foram gerados no mês de maio de 2018, e compilam informações entre os
anos de 2010 e 2017. Os arquivos estão no formato JSON, e seus atributos
iniciais e conteúdos são apresentados a seguir.

\subsubsection{Perfil profissional dos docentes vinculados às
pós-graduações}\label{perfil-profissional-dos-docentes-vinculados-as-pos-graduacoes}

\begin{Shaded}
\begin{Highlighting}[]
\NormalTok{json.perfil <-}\StringTok{ "data/unbpos.profile.json"}
\KeywordTok{file.info}\NormalTok{(json.perfil)}
\end{Highlighting}
\end{Shaded}

\begin{verbatim}
##                              size isdir mode               mtime
## data/unbpos.profile.json 75162725 FALSE  666 2018-09-18 20:39:32
##                                        ctime               atime exe
## data/unbpos.profile.json 2018-09-18 20:39:32 2018-09-19 20:02:59  no
\end{verbatim}

O arquivo data/unbpos.profile.json apresenta dados sobre o perfil de
todos os docentes vinculados a programas de pós-graduação da UnB, entre
2010 e 2017. Esse arquivo foi fornecido pelos docentes responsáveis pela
disciplina.

\subsubsection{Orientações de mestrado e doutorado realizadas pelos
docentes vinculados às
pós-graduações}\label{orientacoes-de-mestrado-e-doutorado-realizadas-pelos-docentes-vinculados-as-pos-graduacoes}

\begin{Shaded}
\begin{Highlighting}[]
\NormalTok{json.advise <-}\StringTok{ "data/unbpos.advise.json"}
\KeywordTok{file.info}\NormalTok{(json.advise)}
\end{Highlighting}
\end{Shaded}

\begin{verbatim}
##                             size isdir mode               mtime
## data/unbpos.advise.json 29828920 FALSE  666 2018-09-18 20:39:31
##                                       ctime               atime exe
## data/unbpos.advise.json 2018-09-18 20:39:31 2018-09-19 20:03:08  no
\end{verbatim}

O arquivo data/unbpos.advise.json apresenta dados sobre o orientações de
mestrado e doutorado feitas por todos os docentes vinculados a programas
de pós-graduação da UnB, entre 2010 e 2017. Esse arquivo foi fornecido
pelos docentes responsáveis pela disciplina.

\subsubsection{Produção bibliográfica gerada pelos docentes vinculados
às
pós-graduações}\label{producao-bibliografica-gerada-pelos-docentes-vinculados-as-pos-graduacoes}

\begin{Shaded}
\begin{Highlighting}[]
\NormalTok{json.producao.bibliografica <-}\StringTok{ "data/unbpos.publication.json"}
\KeywordTok{file.info}\NormalTok{(json.producao.bibliografica) }
\end{Highlighting}
\end{Shaded}

\begin{verbatim}
##                                  size isdir mode               mtime
## data/unbpos.publication.json 33546293 FALSE  666 2018-09-18 20:39:33
##                                            ctime               atime exe
## data/unbpos.publication.json 2018-09-18 20:39:33 2018-09-19 20:03:11  no
\end{verbatim}

O arquivo data/unbpos.publication.json apresenta dados sobre a produção
bibliográfica gerada por todos os docentes vinculados a programas de
pós-graduação da UnB, entre 2010 e 2017.

\subsubsection{Agrupamento dos docentes conforme áreas de
atuação}\label{agrupamento-dos-docentes-conforme-areas-de-atuacao}

\begin{Shaded}
\begin{Highlighting}[]
\NormalTok{json.researchers_by_area <-}\StringTok{ "data/unbpos.researchers_by_area.json"} 
\KeywordTok{file.info}\NormalTok{(json.researchers_by_area)}
\end{Highlighting}
\end{Shaded}

\begin{verbatim}
##                                       size isdir mode               mtime
## data/unbpos.researchers_by_area.json 64366 FALSE  666 2018-09-18 20:39:33
##                                                    ctime
## data/unbpos.researchers_by_area.json 2018-09-18 20:39:33
##                                                    atime exe
## data/unbpos.researchers_by_area.json 2018-09-19 20:03:13  no
\end{verbatim}

O arquivo data/unbpos.researchers\_by\_area.json apresenta as
vinculações de todos os docentes que declararam atuar em cada uma das
áreas de pós-graduação do Sistema Nacional de Pós-Graduação da CAPES,
conforme apresenta-se registrada essa informação no currículo Lattes de
cada um, em data recente.

\begin{Shaded}
\begin{Highlighting}[]
\KeywordTok{file.info}\NormalTok{(}\StringTok{'data/unbpos.graph.json'}\NormalTok{)}
\end{Highlighting}
\end{Shaded}

\begin{verbatim}
##                          size isdir mode               mtime
## data/unbpos.graph.json 503798 FALSE  666 2018-09-18 20:39:31
##                                      ctime               atime exe
## data/unbpos.graph.json 2018-09-18 20:39:31 2018-09-19 20:02:55  no
\end{verbatim}

\subsubsection{Redes de colaboração entre
docentes}\label{redes-de-colaboracao-entre-docentes}

O arquivo data/UnBPosGeral/graph.json apresenta redes de colaboração na
co-autoria de artigos cientpificos, feitas entre os docentes vinculados
a programas de pós-graduação da UnB, entre 2010 e 2017.

\subsection{CRISP-DM Fase.Atividade 2.2 - Descrição dos
Dados}\label{crisp-dm-fase.atividade-2.2---descricao-dos-dados}

Para ler e manipular inicialmente esses dados, serão usadas
primordialmente as bibliotecas seguintes

\begin{Shaded}
\begin{Highlighting}[]
\KeywordTok{library}\NormalTok{(jsonlite)}
\KeywordTok{library}\NormalTok{(listviewer)}
\KeywordTok{library}\NormalTok{(readxl)}
\KeywordTok{library}\NormalTok{(readr)}
\KeywordTok{library}\NormalTok{(readtext)}
\KeywordTok{library}\NormalTok{(ggplot2)}
\KeywordTok{library}\NormalTok{(tidyverse)}
\end{Highlighting}
\end{Shaded}

\begin{verbatim}
## -- Attaching packages -------------------- tidyverse 1.2.1 --
\end{verbatim}

\begin{verbatim}
## v tibble  1.4.2     v dplyr   0.7.6
## v tidyr   0.8.1     v stringr 1.3.1
## v purrr   0.2.5     v forcats 0.3.0
\end{verbatim}

\begin{verbatim}
## -- Conflicts ----------------------- tidyverse_conflicts() --
## x dplyr::filter()  masks stats::filter()
## x purrr::flatten() masks jsonlite::flatten()
## x dplyr::lag()     masks stats::lag()
\end{verbatim}

\begin{Shaded}
\begin{Highlighting}[]
\KeywordTok{library}\NormalTok{(stringr)}
\end{Highlighting}
\end{Shaded}

Como já informado, a descrição dos dados verifica se os dados sendo
acessados terão potencial para responder às questões de \emph{data
mining}. Além disso, deve-se avaliar qual o volume de dados, a estrutura
dos dados (tipos), codificações usadas, etc. Neste projeto, a descrição
dos dados é responsabilidade parcial dos alunos, tendo em vista que esta
seção já oferece uma descrição inicial simplificada. O relatório final
deve conter descrições significativas e aprofundadas dos dados.

\subsubsection{Descrição dos dados do
perfil}\label{descricao-dos-dados-do-perfil}

O arquivo unb.perfis.json, que contém dados que caracterizam o perfil
profissional de todos os docentes do grupo sob análise, podem ser lido
por meio do comando seguinte.

\begin{Shaded}
\begin{Highlighting}[]
\NormalTok{unb.prof <-}\StringTok{ }\KeywordTok{fromJSON}\NormalTok{(}\StringTok{"data/unbpos.profile.json"}\NormalTok{)}
\end{Highlighting}
\end{Shaded}

A quantidade de docentes sob análise é apresentada a seguir.

\begin{Shaded}
\begin{Highlighting}[]
\KeywordTok{length}\NormalTok{(unb.prof)}
\end{Highlighting}
\end{Shaded}

\begin{verbatim}
## [1] 1764
\end{verbatim}

Para gerar uma apresentação inicial dos dados que estão contido nos
dados de perfil dos docentes, pode-se usar a função glimpse, da
biblioteca dplyr, como ilustra o código seguinte, que apresenta os
atributos típicos que podem ser obtidos relativamente a um pesquisador
específico, o mais antigo docente ainda em exercício na UnB a ter criado
seu registro na plataforma Lattes.

\begin{Shaded}
\begin{Highlighting}[]
\KeywordTok{glimpse}\NormalTok{(unb.prof[[}\DecValTok{1}\NormalTok{]], }\DataTypeTok{width =} \DecValTok{30}\NormalTok{)}
\end{Highlighting}
\end{Shaded}

\begin{verbatim}
## List of 7
##  $ nome                  : chr "Norai Romeu Rocco"
##  $ resumo_cv             : chr "Possui graduação em Matemática (licenciatura plena) pela Universidade Estadual Paulista Júlio de Mesquita Filho"| __truncated__
##  $ areas_de_atuacao      :'data.frame':  5 obs. of  4 variables:
##   ..$ grande_area  : chr [1:5] "CIENCIAS_EXATAS_E_DA_TERRA" "CIENCIAS_EXATAS_E_DA_TERRA" "CIENCIAS_EXATAS_E_DA_TERRA" "CIENCIAS_EXATAS_E_DA_TERRA" ...
##   ..$ area         : chr [1:5] "Matemática" "Matemática" "Matemática" "Ciência da Computação" ...
##   ..$ sub_area     : chr [1:5] "" "Álgebra" "Álgebra" "Matemática da Computação" ...
##   ..$ especialidade: chr [1:5] "" "" "Grupos de Álgebra Não-Comutaviva" "Matemática Simbólica" ...
##  $ endereco_profissional :List of 8
##   ..$ instituicao: chr "Universidade de Brasília"
##   ..$ orgao      : chr "Instituto de Ciências Exatas"
##   ..$ unidade    : chr "Departamento de Matemática"
##   ..$ DDD        : chr "061"
##   ..$ telefone   : chr "31076442"
##   ..$ bairro     : chr "Asa Norte"
##   ..$ cep        : chr "70910900"
##   ..$ cidade     : chr "Brasília"
##  $ producao_bibiografica :List of 4
##   ..$ ARTIGO_ACEITO                         :'data.frame':   1 obs. of  10 variables:
##   .. ..$ natureza        : chr "NAO_INFORMADO"
##   .. ..$ titulo          : chr "Finiteness conditions for the non-abelian tensor product of groups"
##   .. ..$ periodico       : chr "MONATSHEFTE FUR MATHEMATIK"
##   .. ..$ ano             : chr "2017"
##   .. ..$ volume          : chr ""
##   .. ..$ issn            : chr "00269255"
##   .. ..$ paginas         : chr " - "
##   .. ..$ doi             : chr "10.1007/s00605-017-1143-x"
##   .. ..$ autores         :List of 1
##   .. ..$ autores-endogeno:List of 1
##   ..$ DEMAIS_TIPOS_DE_PRODUCAO_BIBLIOGRAFICA:'data.frame':   7 obs. of  9 variables:
##   .. ..$ natureza          : chr [1:7] "DIVULGAÇÃO DE RESULTADOS DE PESQUISA" "DIVULGAÇÃO DE RESULTADOS DE PESQUISA" "DIVULGAÇÃO DE RESULTADOS DE PESQUISA" "DIVULGAÇÃO DE RESULTADOS DE PESQUISA" ...
##   .. ..$ titulo            : chr [1:7] "NON-ABELIAN TENSOR SQUARE OF FINITE-BY-NILPOTENT GROUPS" "The q-tensor square of finitely generated nilpotent groups, q >=0" "The q-tensor square of finitely generated nilpotent groups, q >=0" "THE NON-ABELIAN TENSOR SQUARE OF RESIDUALLY FINITE GROUPS" ...
##   .. ..$ ano               : chr [1:7] "2015" "2016" "2016" "2016" ...
##   .. ..$ pais_de_publicacao: chr [1:7] "Estados Unidos" "Estados Unidos" "Estados Unidos" "Estados Unidos" ...
##   .. ..$ editora           : chr [1:7] "" "ArXiv.com - Cornell University Library" "ArXiv.com - Cornell University Library" "ArXiv.com - Cornell University Library" ...
##   .. ..$ doi               : chr [1:7] "" "" "" "" ...
##   .. ..$ numero_de_paginas : chr [1:7] "8" "12" "12" "11" ...
##   .. ..$ autores           :List of 7
##   .. ..$ autores-endogeno  :List of 7
##   ..$ EVENTO                                :'data.frame':   1 obs. of  11 variables:
##   .. ..$ natureza        : chr "RESUMO"
##   .. ..$ titulo          : chr "On Semidirect Products and non-abelian Tensor Products of Groups"
##   .. ..$ nome_do_evento  : chr "XIX Colóquio Latinoamericano de Álgebra"
##   .. ..$ ano_do_trabalho : chr "2012"
##   .. ..$ pais_do_evento  : chr "Chile"
##   .. ..$ cidade_do_evento: chr "Pucón - Chile"
##   .. ..$ doi             : chr ""
##   .. ..$ classificacao   : chr "INTERNACIONAL"
##   .. ..$ paginas         : chr " - "
##   .. ..$ autores         :List of 1
##   .. ..$ autores-endogeno:List of 1
##   ..$ PERIODICO                             :'data.frame':   6 obs. of  10 variables:
##   .. ..$ natureza        : chr [1:6] "COMPLETO" "COMPLETO" "COMPLETO" "COMPLETO" ...
##   .. ..$ titulo          : chr [1:6] "On the q-tensor square of a group" "A survey of non-abelian tensor products of groups and related constructions" "The q-tensor square of finitely generated nilpotent groups, q odd" "Non-abelian tensor square of finite-by-nilpotent groups" ...
##   .. ..$ periodico       : chr [1:6] "Journal of Group Theory" "Boletim da Sociedade Paranaense de Matemática" "JOURNAL OF ALGEBRA AND ITS APPLICATIONS" "Archiv der Mathematik (Printed ed.)" ...
##   .. ..$ ano             : chr [1:6] "2011" "2012" "2016" "2016" ...
##   .. ..$ volume          : chr [1:6] "14" "30" "16" "107" ...
##   .. ..$ issn            : chr [1:6] "14335883" "21751188" "02194988" "0003889X" ...
##   .. ..$ paginas         : chr [1:6] "785 - 805" "77 - 89" "1750211 - " "127 - 133" ...
##   .. ..$ doi             : chr [1:6] "10.1515/JGT.2010.084" "10.5269/bspm.v30i1.13350" "10.1142/S0219498817502115" "10.1007/s00013-016-0930-2" ...
##   .. ..$ autores         :List of 6
##   .. ..$ autores-endogeno:List of 6
##  $ orientacoes_academicas:List of 3
##   ..$ ORIENTACAO_CONCLUIDA_DOUTORADO   :'data.frame':    3 obs. of  13 variables:
##   .. ..$ natureza                   : chr [1:3] "Tese de doutorado" "Tese de doutorado" "Tese de doutorado"
##   .. ..$ titulo                     : chr [1:3] "Cotas superiores para o expoente e o número mínimo de geradores do quadrado q-tensorial de grupos nilpotentes, q geq 0." "Uma Apresentação Policíclica para o Quadrado q-Tensorial de um Grupo Policíclico" "Quadrado Tensorial Não-Abeliano de p-Grupos Finitos com Subgrupo Derivado de Ordem p, p ímpar"
##   .. ..$ ano                        : chr [1:3] "2011" "2011" "2017"
##   .. ..$ id_lattes_aluno            : chr [1:3] "9037151037918091" "8664599889120339" "0723203301483174"
##   .. ..$ nome_aluno                 : chr [1:3] "Eunice Cândida Pereira Rodrigues" "Ivonildes Ribeiro Martins" "Cleilton Aparecido Canal"
##   .. ..$ instituicao                : chr [1:3] "Universidade de Brasília" "Universidade de Brasília" "Universidade de Brasília"
##   .. ..$ curso                      : chr [1:3] "Matemática" "Matemática" "Matemática"
##   .. ..$ codigo_do_curso            : chr [1:3] "51500035" "51500035" "51500035"
##   .. ..$ bolsa                      : chr [1:3] "SIM" "SIM" "NAO"
##   .. ..$ agencia_financiadora       : chr [1:3] "Fundação de Amparo à Pesquisa do Estado de Mato Grosso" "Conselho Nacional de Desenvolvimento Científico e Tecnológico" ""
##   .. ..$ codigo_agencia_financiadora: chr [1:3] "035600000004" "002200000000" ""
##   .. ..$ nome_orientadores          :List of 3
##   .. ..$ id_lattes_orientadores     :List of 3
##   ..$ ORIENTACAO_CONCLUIDA_MESTRADO    :'data.frame':    3 obs. of  13 variables:
##   .. ..$ natureza                   : chr [1:3] "Dissertação de mestrado" "Dissertação de mestrado" "Dissertação de mestrado"
##   .. ..$ titulo                     : chr [1:3] "Algumas Cotas Súperiores para aordem do Quadrado Tensorial não abeliano de um Grupo" "O Grau de Permutabilidade de Subgrupos de um Grupo Finito" "Sobre pE-grupos e pA-grupos finitos"
##   .. ..$ ano                        : chr [1:3] "2010" "2011" "2012"
##   .. ..$ id_lattes_aluno            : chr [1:3] "5367744818899315" "" "0121355793029434"
##   .. ..$ nome_aluno                 : chr [1:3] "Bruno Cesar Rodrigues Lima" "Mônica Aparecida Crunivel Valadão" "Marina Gabriella Ribeiro Bardella"
##   .. ..$ instituicao                : chr [1:3] "Universidade de Brasília" "Universidade de Brasília" "Universidade de Brasília"
##   .. ..$ curso                      : chr [1:3] "Matemática" "Matemática" "Matemática"
##   .. ..$ codigo_do_curso            : chr [1:3] "51500035" "51500035" "51500035"
##   .. ..$ bolsa                      : chr [1:3] "NAO" "SIM" "SIM"
##   .. ..$ agencia_financiadora       : chr [1:3] "" "Coordenação de Aperfeiçoamento de Pessoal de Nível Superior" "Coordenação de Aperfeiçoamento de Pessoal de Nível Superior"
##   .. ..$ codigo_agencia_financiadora: chr [1:3] "" "045000000000" "045000000000"
##   .. ..$ nome_orientadores          :List of 3
##   .. ..$ id_lattes_orientadores     :List of 3
##   ..$ ORIENTACAO_EM_ANDAMENTO_DOUTORADO:'data.frame':    1 obs. of  13 variables:
##   .. ..$ natureza                   : chr "Tese de doutorado"
##   .. ..$ titulo                     : chr "Quadrado Tensorial não Abeliano de certas classes de Grupos Finitos"
##   .. ..$ ano                        : chr "2014"
##   .. ..$ id_lattes_aluno            : chr "1933036212945705"
##   .. ..$ nome_aluno                 : chr "Juliana Silva Canella"
##   .. ..$ instituicao                : chr "Universidade de Brasília"
##   .. ..$ curso                      : chr "Matemática"
##   .. ..$ codigo_do_curso            : chr "51500035"
##   .. ..$ bolsa                      : chr "SIM"
##   .. ..$ agencia_financiadora       : chr "Coordenação de Aperfeiçoamento de Pessoal de Nível Superior"
##   .. ..$ codigo_agencia_financiadora: chr "045000000000"
##   .. ..$ nome_orientadores          :List of 1
##   .. ..$ id_lattes_orientadores     :List of 1
##  $ senioridade           : chr "8"
\end{verbatim}

Uma breve inspeção visual dos atributos anteriormente apresentados
permite inferir que o pesquisador sob análise:

\begin{itemize}
\tightlist
\item
  Atua predominantemente na área de matemática.
\item
  Trabalha no Instituto de Ciências Exatas da UnB.
\item
  Possui três artigos recentes publicados, além de um aceito para
  publicação.
\item
  Possui uma orientação de doutorado em andamento, iniciada em 2014.
\item
  Foi classificado com senioridade 5.
\end{itemize}

\paragraph{Potencial de utilização dos dados do perfil dos
docentes}\label{potencial-de-utilizacao-dos-dados-do-perfil-dos-docentes}

Esses dados terão potencial para responder às questões de \emph{data
mining}? O que é possível gerar a partir desses dados, para o conjunto
dos 1592 docentes da UnB? A fim de compreender a relevância dos dados
para a avaliação da produção acadêmica nas pós-graduações brasileiras
pode-se recorrer a trabalhos como os seguintes:

\begin{itemize}
\tightlist
\item
  Leite (2018) apresenta, em suas ``Considerações básicas sobre a
  Avaliação do Sistema Nacional de Pós-Graduação'', o conjunto dos itens
  que são tópicos de avaliação das pós-graduações pela CAPES, e que
  envolvem, entre outros:

  \begin{itemize}
  \tightlist
  \item
    Avaliação do corpo docente, com 20\% a 30\% de peso na avaliação
    total do programa, a depender do seu tipo. Analisando-se de forma
    mais detalhada os critérios de avaliação do corpo docente, indicados
    por Leite, o que é possível gerar com base nos dados disponíveis em
    unb.prof? Há dados que permitam identificar o perfil do docente,
    como proposto pela CAPES, inclusive no documento de área específica
    na qual atua o pesquisador? Que outros aspectos relevantes para a
    CAPES podem ser levantados com base nos dados dessa fonte?
  \item
    Avaliação do corpo discente, Teses e dissertações, com 30\% a 20\%
    de peso na avaliação total do programa, a depender de seu tipo. Os
    dados sobre orientação permitem fazer quais tipos de avaliações do
    corpo discente?
  \item
    Avaliação da produção intelectual, com 40\% de peso na avaliação
    total. Qual a relevância dos dados em unb.prof para essa avaliação?
    Que outros arquivos podem melhor subsidiar essa avaliação?
  \end{itemize}
\item
  Em busca de considerar outros fatores relevantes para a avaliação da
  pós-graduação, não considerados no modelo da CAPES, pode-se recorrer
  ao trabalho de Kalpazidou Schmidt e Graversen (2018), que apresentam
  um conjunto de fatores persistentes que facilitam a existência de
  ambientes de pesquisa inovadores e dinâmicos, dentre os quais se
  destaca:

  \begin{itemize}
  \tightlist
  \item
    Atividade em pesquisas com elevado grau de impacto social;
  \item
    Promoção de elevado grau de autonomia individual, tanto do ponto de
    vista teórico quanto metodológico;
  \item
    Possuem um bom clima de trabalho, baseado no trabalho em times;
  \item
    São internacioinalmente bem conhecidas etc.
  \end{itemize}

  Estariam esses fatores contemplados, de alguma forma, memso que
  parcialmente, nos dados presentes em unb.prof? Ou em qualquer outros
  dos arquivos? Cabe explorar.
\end{itemize}

\subsubsection{Descrição dos dados de
orientações}\label{descricao-dos-dados-de-orientacoes}

\begin{Shaded}
\begin{Highlighting}[]
\NormalTok{unb.adv <-}\StringTok{ }\KeywordTok{fromJSON}\NormalTok{(}\StringTok{"data/unbpos.advise.json"}\NormalTok{)}
\KeywordTok{names}\NormalTok{(unb.adv)}
\end{Highlighting}
\end{Shaded}

\begin{verbatim}
## [1] "ORIENTACAO_EM_ANDAMENTO_DE_POS_DOUTORADO"    
## [2] "ORIENTACAO_EM_ANDAMENTO_DOUTORADO"           
## [3] "ORIENTACAO_EM_ANDAMENTO_MESTRADO"            
## [4] "ORIENTACAO_EM_ANDAMENTO_GRADUACAO"           
## [5] "ORIENTACAO_EM_ANDAMENTO_INICIACAO_CIENTIFICA"
## [6] "ORIENTACAO_CONCLUIDA_POS_DOUTORADO"          
## [7] "ORIENTACAO_CONCLUIDA_DOUTORADO"              
## [8] "ORIENTACAO_CONCLUIDA_MESTRADO"               
## [9] "OUTRAS_ORIENTACOES_CONCLUIDAS"
\end{verbatim}

\begin{Shaded}
\begin{Highlighting}[]
\KeywordTok{names}\NormalTok{(unb.adv}\OperatorTok{$}\NormalTok{ORIENTACAO_CONCLUIDA_DOUTORADO)}
\end{Highlighting}
\end{Shaded}

\begin{verbatim}
## [1] "2010" "2011" "2012" "2013" "2014" "2015" "2016" "2017"
\end{verbatim}

\begin{Shaded}
\begin{Highlighting}[]
\KeywordTok{length}\NormalTok{(unb.adv}\OperatorTok{$}\NormalTok{ORIENTACAO_CONCLUIDA_DOUTORADO}\OperatorTok{$}\StringTok{`}\DataTypeTok{2016}\StringTok{`}\OperatorTok{$}\NormalTok{natureza)}
\end{Highlighting}
\end{Shaded}

\begin{verbatim}
## [1] 606
\end{verbatim}

\begin{Shaded}
\begin{Highlighting}[]
\KeywordTok{head}\NormalTok{(}\KeywordTok{sort}\NormalTok{(}\KeywordTok{table}\NormalTok{(unb.adv}\OperatorTok{$}\NormalTok{ORIENTACAO_CONCLUIDA_DOUTORADO}\OperatorTok{$}\StringTok{`}\DataTypeTok{2017}\StringTok{`}\OperatorTok{$}\NormalTok{curso), }\DataTypeTok{decreasing =} \OtherTok{TRUE}\NormalTok{), }\DecValTok{10}\NormalTok{)}
\end{Highlighting}
\end{Shaded}

\begin{verbatim}
## 
##                           Ciências da Saúde 
##                                          17 
##                                   Geografia 
##                                          15 
##                                    Educação 
##                                          14 
##                Psicologia Clínica e Cultura 
##                                          14 
## Processos de Desenvolvimento Humano e Saúde 
##                                          13 
##                                    Economia 
##                                          12 
##                                   Geotecnia 
##                                          11 
##                             Biologia Animal 
##                                          10 
##                      Ciências da Informação 
##                                          10 
##                                    Geologia 
##                                          10
\end{verbatim}

\begin{Shaded}
\begin{Highlighting}[]
\KeywordTok{head}\NormalTok{(}\KeywordTok{sort}\NormalTok{(}\KeywordTok{table}\NormalTok{(unb.adv}\OperatorTok{$}\NormalTok{ORIENTACAO_CONCLUIDA_MESTRADO}\OperatorTok{$}\StringTok{`}\DataTypeTok{2017}\StringTok{`}\OperatorTok{$}\NormalTok{curso), }\DataTypeTok{decreasing =} \OtherTok{TRUE}\NormalTok{), }\DecValTok{10}\NormalTok{)}
\end{Highlighting}
\end{Shaded}

\begin{verbatim}
## 
##                                    Economia 
##                                          43 
##                                     Direito 
##                                          34 
##                           Ciências da Saúde 
##                                          28 
##             Ciências e Tecnologias em Saúde 
##                                          26 
##               Estruturas e Construção Civil 
##                                          25 
##                         Estudos de Tradução 
##                                          25 
##                                  Literatura 
##                                          21 
## Processos de Desenvolvimento Humano e Saúde 
##                                          20 
##                                    Geologia 
##                                          19 
##                          Ciências Mecânicas 
##                                          18
\end{verbatim}

\subsubsection{Descrição dos dados de produção
bibliográfica}\label{descricao-dos-dados-de-producao-bibliografica}

\begin{Shaded}
\begin{Highlighting}[]
\NormalTok{unb.pub <-}\StringTok{ }\KeywordTok{fromJSON}\NormalTok{(}\StringTok{"data/unbpos.publication.json"}\NormalTok{)}
\KeywordTok{names}\NormalTok{(unb.pub)}
\end{Highlighting}
\end{Shaded}

\begin{verbatim}
## [1] "PERIODICO"                             
## [2] "LIVRO"                                 
## [3] "CAPITULO_DE_LIVRO"                     
## [4] "TEXTO_EM_JORNAIS"                      
## [5] "EVENTO"                                
## [6] "ARTIGO_ACEITO"                         
## [7] "DEMAIS_TIPOS_DE_PRODUCAO_BIBLIOGRAFICA"
\end{verbatim}

\begin{Shaded}
\begin{Highlighting}[]
\KeywordTok{names}\NormalTok{(unb.pub}\OperatorTok{$}\NormalTok{PERIODICO}\OperatorTok{$}\StringTok{`}\DataTypeTok{2012}\StringTok{`}\NormalTok{)}
\end{Highlighting}
\end{Shaded}

\begin{verbatim}
##  [1] "natureza"         "titulo"           "periodico"       
##  [4] "ano"              "volume"           "issn"            
##  [7] "paginas"          "doi"              "autores"         
## [10] "autores-endogeno"
\end{verbatim}

\begin{Shaded}
\begin{Highlighting}[]
\KeywordTok{head}\NormalTok{(}\KeywordTok{sort}\NormalTok{(}\KeywordTok{table}\NormalTok{(unb.pub}\OperatorTok{$}\NormalTok{PERIODICO}\OperatorTok{$}\StringTok{`}\DataTypeTok{2017}\StringTok{`}\OperatorTok{$}\NormalTok{periodico), }\DataTypeTok{decreasing =} \OtherTok{TRUE}\NormalTok{), }\DecValTok{10}\NormalTok{)}
\end{Highlighting}
\end{Shaded}

\begin{verbatim}
## 
##               REVISTA DE SAÚDE PÚBLICA (ONLINE) 
##                                              23 
##                                        PLoS One 
##                                              21 
##                              ESPACIOS (CARACAS) 
##                                              16 
##                              Scientific Reports 
##                                              16 
##                        Ciencia & Saude Coletiva 
##                                              15 
##                 GENETICS AND MOLECULAR RESEARCH 
##                                              15 
##                          CADERNOS DE PROSPECÇÃO 
##                                              14 
##        JOURNAL OF SOUTH AMERICAN EARTH SCIENCES 
##                                              14 
##           Journal of Molecular Modeling (Print) 
##                                              13 
## RBC. REVISTA BRASILEIRA DE CARTOGRAFIA (ONLINE) 
##                                              13
\end{verbatim}

\begin{Shaded}
\begin{Highlighting}[]
\KeywordTok{head}\NormalTok{(}\KeywordTok{sort}\NormalTok{(}\KeywordTok{table}\NormalTok{(unb.pub}\OperatorTok{$}\NormalTok{LIVRO}\OperatorTok{$}\StringTok{`}\DataTypeTok{2015}\StringTok{`}\OperatorTok{$}\NormalTok{nome_da_editora), }\DataTypeTok{decreasing =} \OtherTok{TRUE}\NormalTok{), }\DecValTok{10}\NormalTok{)}
\end{Highlighting}
\end{Shaded}

\begin{verbatim}
## 
##                            ANPOF                                  
##                               23                               13 
##         Novas Edições Acadêmicas            Laccademia Publishing 
##                               12                                8 
## Editora Universidade de Brasília                  Pontes Editores 
##                                7                                7 
##                      Lumen Juris                       Fino Traço 
##                                6                                5 
##                         INEP/MEC                              LTr 
##                                5                                5
\end{verbatim}

\subsubsection{Descrição dos dados de agregação de docentes por
área}\label{descricao-dos-dados-de-agregacao-de-docentes-por-area}

\begin{Shaded}
\begin{Highlighting}[]
\NormalTok{unb.area <-}\StringTok{ }\KeywordTok{fromJSON}\NormalTok{(}\StringTok{"data/unbpos.researchers_by_area.json"}\NormalTok{)}
\NormalTok{unb.area.df <-}\StringTok{ }\KeywordTok{cbind}\NormalTok{(}\KeywordTok{names}\NormalTok{(unb.area}\OperatorTok{$}\StringTok{`}\DataTypeTok{Areas dos pesquisadores}\StringTok{`}\NormalTok{),}
\NormalTok{           (}\KeywordTok{sapply}\NormalTok{(unb.area}\OperatorTok{$}\StringTok{`}\DataTypeTok{Areas dos pesquisadores}\StringTok{`}\NormalTok{, }\ControlFlowTok{function}\NormalTok{(x) }\KeywordTok{length}\NormalTok{(x))))}
\KeywordTok{rownames}\NormalTok{(unb.area.df) <-}\StringTok{ }\KeywordTok{c}\NormalTok{(}\DecValTok{1}\OperatorTok{:}\KeywordTok{nrow}\NormalTok{(unb.area.df)); }\KeywordTok{colnames}\NormalTok{(unb.area.df) <-}\StringTok{ }\KeywordTok{c}\NormalTok{(}\StringTok{"Area"}\NormalTok{, }\StringTok{"Professores"}\NormalTok{)}
\KeywordTok{glimpse}\NormalTok{(unb.area.df)}
\end{Highlighting}
\end{Shaded}

\begin{verbatim}
##  chr [1:85, 1:2] "Administração" "Agronomia" "Antropologia" ...
##  - attr(*, "dimnames")=List of 2
##   ..$ : chr [1:85] "1" "2" "3" "4" ...
##   ..$ : chr [1:2] "Area" "Professores"
\end{verbatim}

\begin{Shaded}
\begin{Highlighting}[]
\KeywordTok{head}\NormalTok{(unb.area.df[])}
\end{Highlighting}
\end{Shaded}

\begin{verbatim}
##   Area                      Professores
## 1 "Administração"           "101"      
## 2 "Agronomia"               "65"       
## 3 "Antropologia"            "52"       
## 4 "Arqueologia"             "2"        
## 5 "Arquitetura e Urbanismo" "42"       
## 6 "Artes"                   "85"
\end{verbatim}

\subsubsection{Descrição dos dados de redes de
colaboração}\label{descricao-dos-dados-de-redes-de-colaboracao}

\subsection{CRISP-DM Fase.Atividade 2.3 - Análise exploratória dos
dados}\label{crisp-dm-fase.atividade-2.3---analise-exploratoria-dos-dados}

Como já informado, a análise exploratória dos dados possibilita um
entendimento mais profundo da relação estatística existente entre os
dados dos \emph{datasets} para um melhor entendimento da qualidade
daqueles dados para os objetivos do projeto.

Como já informado, a análise exploratória dos dados é responsabilidade
parcial dos alunos, tendo em vista que este relatório apresenta uma
análise exploratória preliminar. O relatório final deve conter análises
exploratórias dos dados que sejam significativas e aprofundadas.

\subsubsection{Arquivo Profile}\label{arquivo-profile}

\begin{Shaded}
\begin{Highlighting}[]
\CommentTok{# jsonedit(unb.prof)}
\CommentTok{# Número de áreas de atuação cumulativo}
\KeywordTok{sum}\NormalTok{(}\KeywordTok{sapply}\NormalTok{(unb.prof, }\ControlFlowTok{function}\NormalTok{(x) }\KeywordTok{nrow}\NormalTok{(x}\OperatorTok{$}\NormalTok{areas_de_atuacao)))}
\end{Highlighting}
\end{Shaded}

\begin{verbatim}
## [1] 7119
\end{verbatim}

\begin{Shaded}
\begin{Highlighting}[]
\CommentTok{# Número de áreas de atuação por pessoa}
\KeywordTok{table}\NormalTok{(}\KeywordTok{unlist}\NormalTok{(}\KeywordTok{sapply}\NormalTok{(unb.prof, }\ControlFlowTok{function}\NormalTok{(x) }\KeywordTok{nrow}\NormalTok{(x}\OperatorTok{$}\NormalTok{areas_de_atuacao))))}
\end{Highlighting}
\end{Shaded}

\begin{verbatim}
## 
##   1   2   3   4   5   6  10 
## 119 205 350 331 342 416   1
\end{verbatim}

\begin{Shaded}
\begin{Highlighting}[]
\CommentTok{# Número de pessoas por grande area}
\KeywordTok{table}\NormalTok{(}\KeywordTok{unlist}\NormalTok{(}\KeywordTok{sapply}\NormalTok{(unb.prof, }\ControlFlowTok{function}\NormalTok{(x) (x}\OperatorTok{$}\NormalTok{areas_de_atuacao}\OperatorTok{$}\NormalTok{grande_area))))}
\end{Highlighting}
\end{Shaded}

\begin{verbatim}
## 
##                                     CIENCIAS_AGRARIAS 
##                         27                        427 
##        CIENCIAS_BIOLOGICAS          CIENCIAS_DA_SAUDE 
##                        780                        601 
## CIENCIAS_EXATAS_E_DA_TERRA           CIENCIAS_HUMANAS 
##                       1075                       1641 
## CIENCIAS_SOCIAIS_APLICADAS                ENGENHARIAS 
##                       1172                        697 
## LINGUISTICA_LETRAS_E_ARTES                     OUTROS 
##                        634                         65
\end{verbatim}

\begin{Shaded}
\begin{Highlighting}[]
\CommentTok{# Número de pessoas que produziram os específicos tipos de produção}
\KeywordTok{table}\NormalTok{(}\KeywordTok{unlist}\NormalTok{(}\KeywordTok{sapply}\NormalTok{(unb.prof, }\ControlFlowTok{function}\NormalTok{(x) }\KeywordTok{names}\NormalTok{(x}\OperatorTok{$}\NormalTok{producao_bibiografica))))}
\end{Highlighting}
\end{Shaded}

\begin{verbatim}
## 
##                          ARTIGO_ACEITO 
##                                    349 
##                      CAPITULO_DE_LIVRO 
##                                   1346 
## DEMAIS_TIPOS_DE_PRODUCAO_BIBLIOGRAFICA 
##                                    532 
##                                 EVENTO 
##                                   1504 
##                                  LIVRO 
##                                    867 
##                              PERIODICO 
##                                   1716 
##                       TEXTO_EM_JORNAIS 
##                                    492
\end{verbatim}

\begin{Shaded}
\begin{Highlighting}[]
\CommentTok{# Número de publicações por tipo}
\KeywordTok{sum}\NormalTok{(}\KeywordTok{sapply}\NormalTok{(unb.prof, }\ControlFlowTok{function}\NormalTok{(x) }\KeywordTok{length}\NormalTok{(x}\OperatorTok{$}\NormalTok{producao_bibiografica}\OperatorTok{$}\NormalTok{ARTIGO_ACEITO}\OperatorTok{$}\NormalTok{ano)))}
\end{Highlighting}
\end{Shaded}

\begin{verbatim}
## [1] 563
\end{verbatim}

\begin{Shaded}
\begin{Highlighting}[]
\KeywordTok{sum}\NormalTok{(}\KeywordTok{sapply}\NormalTok{(unb.prof, }\ControlFlowTok{function}\NormalTok{(x) }\KeywordTok{length}\NormalTok{(x}\OperatorTok{$}\NormalTok{producao_bibiografica}\OperatorTok{$}\NormalTok{CAPITULO_DE_LIVRO}\OperatorTok{$}\NormalTok{ano)))}
\end{Highlighting}
\end{Shaded}

\begin{verbatim}
## [1] 8816
\end{verbatim}

\begin{Shaded}
\begin{Highlighting}[]
\KeywordTok{sum}\NormalTok{(}\KeywordTok{sapply}\NormalTok{(unb.prof, }\ControlFlowTok{function}\NormalTok{(x) }\KeywordTok{length}\NormalTok{(x}\OperatorTok{$}\NormalTok{producao_bibiografica}\OperatorTok{$}\NormalTok{LIVRO}\OperatorTok{$}\NormalTok{ano)))}
\end{Highlighting}
\end{Shaded}

\begin{verbatim}
## [1] 2932
\end{verbatim}

\begin{Shaded}
\begin{Highlighting}[]
\KeywordTok{sum}\NormalTok{(}\KeywordTok{sapply}\NormalTok{(unb.prof, }\ControlFlowTok{function}\NormalTok{(x) }\KeywordTok{length}\NormalTok{(x}\OperatorTok{$}\NormalTok{producao_bibiografica}\OperatorTok{$}\NormalTok{PERIODICO}\OperatorTok{$}\NormalTok{ano)))}
\end{Highlighting}
\end{Shaded}

\begin{verbatim}
## [1] 30352
\end{verbatim}

\begin{Shaded}
\begin{Highlighting}[]
\KeywordTok{sum}\NormalTok{(}\KeywordTok{sapply}\NormalTok{(unb.prof, }\ControlFlowTok{function}\NormalTok{(x) }\KeywordTok{length}\NormalTok{(x}\OperatorTok{$}\NormalTok{producao_bibiografica}\OperatorTok{$}\NormalTok{TEXTO_EM_JORNAIS}\OperatorTok{$}\NormalTok{ano)))}
\end{Highlighting}
\end{Shaded}

\begin{verbatim}
## [1] 3042
\end{verbatim}

\begin{Shaded}
\begin{Highlighting}[]
\CommentTok{# Número de pessoas por quantitativo de produções por pessoa 0 = 1; 1 = 2...}
\KeywordTok{table}\NormalTok{(}\KeywordTok{unlist}\NormalTok{(}\KeywordTok{sapply}\NormalTok{(unb.prof, }\ControlFlowTok{function}\NormalTok{(x) }\KeywordTok{length}\NormalTok{(x}\OperatorTok{$}\NormalTok{producao_bibiografica}\OperatorTok{$}\NormalTok{ARTIGO_ACEITO}\OperatorTok{$}\NormalTok{ano))))}
\end{Highlighting}
\end{Shaded}

\begin{verbatim}
## 
##    0    1    2    3    4    5    6    7    9   10   11   15 
## 1415  242   66   21    8    2    3    3    1    1    1    1
\end{verbatim}

\begin{Shaded}
\begin{Highlighting}[]
\KeywordTok{table}\NormalTok{(}\KeywordTok{unlist}\NormalTok{(}\KeywordTok{sapply}\NormalTok{(unb.prof, }\ControlFlowTok{function}\NormalTok{(x) }\KeywordTok{length}\NormalTok{(x}\OperatorTok{$}\NormalTok{producao_bibiografica}\OperatorTok{$}\NormalTok{CAPITULO_DE_LIVRO}\OperatorTok{$}\NormalTok{ano))))}
\end{Highlighting}
\end{Shaded}

\begin{verbatim}
## 
##   0   1   2   3   4   5   6   7   8   9  10  11  12  13  14  15  16  17 
## 418 258 204 139 130  99  68  64  52  43  32  36  31  33  19  14  10  11 
##  18  19  20  21  22  23  24  25  26  27  28  29  30  31  32  33  34  37 
##  13   9  11   6   7   5   6   4   2   7   4   1   3   4   2   3   3   1 
##  38  39  40  44  47  52  56  69 
##   3   1   2   1   1   1   2   1
\end{verbatim}

\begin{Shaded}
\begin{Highlighting}[]
\KeywordTok{table}\NormalTok{(}\KeywordTok{unlist}\NormalTok{(}\KeywordTok{sapply}\NormalTok{(unb.prof, }\ControlFlowTok{function}\NormalTok{(x) }\KeywordTok{length}\NormalTok{(x}\OperatorTok{$}\NormalTok{producao_bibiografica}\OperatorTok{$}\NormalTok{LIVRO}\OperatorTok{$}\NormalTok{ano))))}
\end{Highlighting}
\end{Shaded}

\begin{verbatim}
## 
##   0   1   2   3   4   5   6   7   8   9  10  11  12  14  15  16  18  19 
## 897 322 189 100  67  50  29  29  19  14  12   7   8   2   2   3   2   3 
##  20  21  26  28  31  32  39  49 
##   1   2   1   1   1   1   1   1
\end{verbatim}

\begin{Shaded}
\begin{Highlighting}[]
\KeywordTok{table}\NormalTok{(}\KeywordTok{unlist}\NormalTok{(}\KeywordTok{sapply}\NormalTok{(unb.prof, }\ControlFlowTok{function}\NormalTok{(x) }\KeywordTok{length}\NormalTok{(x}\OperatorTok{$}\NormalTok{producao_bibiografica}\OperatorTok{$}\NormalTok{PERIODICO}\OperatorTok{$}\NormalTok{ano))))}
\end{Highlighting}
\end{Shaded}

\begin{verbatim}
## 
##   0   1   2   3   4   5   6   7   8   9  10  11  12  13  14  15  16  17 
##  48  55  63  74  74 101  83  86  66  75  69  61  60  53  46  47  48  27 
##  18  19  20  21  22  23  24  25  26  27  28  29  30  31  32  33  34  35 
##  34  44  32  38  31  32  20  25  19  26  22  17  18  13  13  11  21  10 
##  36  37  38  39  40  41  42  43  44  45  46  47  48  49  50  51  52  53 
##  10  11  13   6   9   8   7   6   7  14   5   8   5   4   1   6   9   2 
##  54  55  56  57  58  59  60  61  62  63  64  66  67  68  69  70  71  73 
##   1   3   2   3   3   3   2   4   2   6   2   3   2   3   1   1   4   2 
##  74  75  76  77  78  83  86  88  89  90 103 104 126 146 222 233 
##   1   1   1   4   3   1   2   1   2   2   1   1   1   1   1   1
\end{verbatim}

\begin{Shaded}
\begin{Highlighting}[]
\KeywordTok{table}\NormalTok{(}\KeywordTok{unlist}\NormalTok{(}\KeywordTok{sapply}\NormalTok{(unb.prof, }\ControlFlowTok{function}\NormalTok{(x) }\KeywordTok{length}\NormalTok{(x}\OperatorTok{$}\NormalTok{producao_bibiografica}\OperatorTok{$}\NormalTok{TEXTO_EM_JORNAIS}\OperatorTok{$}\NormalTok{ano))))}
\end{Highlighting}
\end{Shaded}

\begin{verbatim}
## 
##    0    1    2    3    4    5    6    7    8    9   10   11   12   14   15 
## 1272  219   92   46   25   20    9    9   10    4    9    4    5    5    3 
##   17   18   19   21   22   23   25   27   29   30   31   34   35   38   39 
##    1    5    1    1    1    1    1    1    3    2    1    1    1    1    1 
##   49   51   67   86  109  146  148  176  178  181 
##    1    1    1    1    1    1    1    1    1    1
\end{verbatim}

\begin{Shaded}
\begin{Highlighting}[]
\CommentTok{# Número de produções por ano}
\KeywordTok{table}\NormalTok{(}\KeywordTok{unlist}\NormalTok{(}\KeywordTok{sapply}\NormalTok{(unb.prof, }\ControlFlowTok{function}\NormalTok{(x) (x}\OperatorTok{$}\NormalTok{producao_bibiografica}\OperatorTok{$}\NormalTok{ARTIGO_ACEITO}\OperatorTok{$}\NormalTok{ano))))}
\end{Highlighting}
\end{Shaded}

\begin{verbatim}
## 
## 2010 2011 2012 2013 2014 2015 2016 2017 
##   21   17   29   44   48   60   93  251
\end{verbatim}

\begin{Shaded}
\begin{Highlighting}[]
\KeywordTok{table}\NormalTok{(}\KeywordTok{unlist}\NormalTok{(}\KeywordTok{sapply}\NormalTok{(unb.prof, }\ControlFlowTok{function}\NormalTok{(x) (x}\OperatorTok{$}\NormalTok{producao_bibiografica}\OperatorTok{$}\NormalTok{CAPITULO_DE_LIVRO}\OperatorTok{$}\NormalTok{ano))))}
\end{Highlighting}
\end{Shaded}

\begin{verbatim}
## 
## 2010 2011 2012 2013 2014 2015 2016 2017 
## 1042 1052 1325  891 1135 1185 1162 1024
\end{verbatim}

\begin{Shaded}
\begin{Highlighting}[]
\KeywordTok{table}\NormalTok{(}\KeywordTok{unlist}\NormalTok{(}\KeywordTok{sapply}\NormalTok{(unb.prof, }\ControlFlowTok{function}\NormalTok{(x) (x}\OperatorTok{$}\NormalTok{producao_bibiografica}\OperatorTok{$}\NormalTok{LIVRO}\OperatorTok{$}\NormalTok{ano))))}
\end{Highlighting}
\end{Shaded}

\begin{verbatim}
## 
## 2010 2011 2012 2013 2014 2015 2016 2017 
##  353  301  373  383  388  426  371  337
\end{verbatim}

\begin{Shaded}
\begin{Highlighting}[]
\KeywordTok{table}\NormalTok{(}\KeywordTok{unlist}\NormalTok{(}\KeywordTok{sapply}\NormalTok{(unb.prof, }\ControlFlowTok{function}\NormalTok{(x) (x}\OperatorTok{$}\NormalTok{producao_bibiografica}\OperatorTok{$}\NormalTok{PERIODICO}\OperatorTok{$}\NormalTok{ano))))}
\end{Highlighting}
\end{Shaded}

\begin{verbatim}
## 
## 2010 2011 2012 2013 2014 2015 2016 2017 
## 3097 3193 3633 3859 3943 4154 4322 4151
\end{verbatim}

\begin{Shaded}
\begin{Highlighting}[]
\KeywordTok{table}\NormalTok{(}\KeywordTok{unlist}\NormalTok{(}\KeywordTok{sapply}\NormalTok{(unb.prof, }\ControlFlowTok{function}\NormalTok{(x) (x}\OperatorTok{$}\NormalTok{producao_bibiografica}\OperatorTok{$}\NormalTok{TEXTO_EM_JORNAIS}\OperatorTok{$}\NormalTok{ano))))}
\end{Highlighting}
\end{Shaded}

\begin{verbatim}
## 
## 2010 2011 2012 2013 2014 2015 2016 2017 
##  459  440  424  374  384  310  360  291
\end{verbatim}

\begin{Shaded}
\begin{Highlighting}[]
\CommentTok{# Número de pessoas que realizaram diferentes tipos de orientações}
\KeywordTok{length}\NormalTok{(}\KeywordTok{unlist}\NormalTok{(}\KeywordTok{sapply}\NormalTok{(unb.prof, }\ControlFlowTok{function}\NormalTok{(x) }\KeywordTok{names}\NormalTok{(x}\OperatorTok{$}\NormalTok{orientacoes_academicas))))}
\end{Highlighting}
\end{Shaded}

\begin{verbatim}
## [1] 7317
\end{verbatim}

\begin{Shaded}
\begin{Highlighting}[]
\CommentTok{# Número de pessoas por tipo de orientação}
\KeywordTok{table}\NormalTok{(}\KeywordTok{unlist}\NormalTok{(}\KeywordTok{sapply}\NormalTok{(unb.prof, }\ControlFlowTok{function}\NormalTok{(x) }\KeywordTok{names}\NormalTok{(x}\OperatorTok{$}\NormalTok{orientacoes_academicas))))}
\end{Highlighting}
\end{Shaded}

\begin{verbatim}
## 
##               ORIENTACAO_CONCLUIDA_DOUTORADO 
##                                          935 
##                ORIENTACAO_CONCLUIDA_MESTRADO 
##                                         1546 
##           ORIENTACAO_CONCLUIDA_POS_DOUTORADO 
##                                          267 
##            ORIENTACAO_EM_ANDAMENTO_DOUTORADO 
##                                         1061 
##            ORIENTACAO_EM_ANDAMENTO_GRADUACAO 
##                                          195 
## ORIENTACAO_EM_ANDAMENTO_INICIACAO_CIENTIFICA 
##                                          591 
##             ORIENTACAO_EM_ANDAMENTO_MESTRADO 
##                                         1168 
##                OUTRAS_ORIENTACOES_CONCLUIDAS 
##                                         1554
\end{verbatim}

\begin{Shaded}
\begin{Highlighting}[]
\CommentTok{#Número de orientações concluidas}
\KeywordTok{sum}\NormalTok{(}\KeywordTok{sapply}\NormalTok{(unb.prof, }\ControlFlowTok{function}\NormalTok{(x) }\KeywordTok{length}\NormalTok{(x}\OperatorTok{$}\NormalTok{orientacoes_academicas}\OperatorTok{$}\NormalTok{ORIENTACAO_CONCLUIDA_MESTRADO}\OperatorTok{$}\NormalTok{ano)))}
\end{Highlighting}
\end{Shaded}

\begin{verbatim}
## [1] 10875
\end{verbatim}

\begin{Shaded}
\begin{Highlighting}[]
\KeywordTok{sum}\NormalTok{(}\KeywordTok{sapply}\NormalTok{(unb.prof, }\ControlFlowTok{function}\NormalTok{(x) }\KeywordTok{length}\NormalTok{(x}\OperatorTok{$}\NormalTok{orientacoes_academicas}\OperatorTok{$}\NormalTok{ORIENTACAO_CONCLUIDA_DOUTORADO}\OperatorTok{$}\NormalTok{ano)))}
\end{Highlighting}
\end{Shaded}

\begin{verbatim}
## [1] 3899
\end{verbatim}

\begin{Shaded}
\begin{Highlighting}[]
\KeywordTok{sum}\NormalTok{(}\KeywordTok{sapply}\NormalTok{(unb.prof, }\ControlFlowTok{function}\NormalTok{(x) }\KeywordTok{length}\NormalTok{(x}\OperatorTok{$}\NormalTok{orientacoes_academicas}\OperatorTok{$}\NormalTok{ORIENTACAO_CONCLUIDA_POS_DOUTORADO}\OperatorTok{$}\NormalTok{ano)))}
\end{Highlighting}
\end{Shaded}

\begin{verbatim}
## [1] 740
\end{verbatim}

\begin{Shaded}
\begin{Highlighting}[]
\CommentTok{# Número de pessoas por quantitativo de orientações por pessoa 0 = 1; 1 = 2...}
\KeywordTok{table}\NormalTok{(}\KeywordTok{unlist}\NormalTok{(}\KeywordTok{sapply}\NormalTok{(unb.prof, }\ControlFlowTok{function}\NormalTok{(x) }\KeywordTok{length}\NormalTok{(x}\OperatorTok{$}\NormalTok{orientacoes_academicas}\OperatorTok{$}\NormalTok{ORIENTACAO_CONCLUIDA_MESTRADO}\OperatorTok{$}\NormalTok{ano))))}
\end{Highlighting}
\end{Shaded}

\begin{verbatim}
## 
##   0   1   2   3   4   5   6   7   8   9  10  11  12  13  14  15  16  17 
## 218 142 126 137 153 139 134 105 113  92  95  76  45  37  42  25  23   6 
##  18  19  20  21  22  23  24  25  26  27  28  30  31  34  38  41  45 
##  15  12   4   2   4   4   2   2   1   3   1   1   1   1   1   1   1
\end{verbatim}

\begin{Shaded}
\begin{Highlighting}[]
\KeywordTok{table}\NormalTok{(}\KeywordTok{unlist}\NormalTok{(}\KeywordTok{sapply}\NormalTok{(unb.prof, }\ControlFlowTok{function}\NormalTok{(x) }\KeywordTok{length}\NormalTok{(x}\OperatorTok{$}\NormalTok{orientacoes_academicas}\OperatorTok{$}\NormalTok{ORIENTACAO_CONCLUIDA_DOUTORADO}\OperatorTok{$}\NormalTok{ano))))}
\end{Highlighting}
\end{Shaded}

\begin{verbatim}
## 
##   0   1   2   3   4   5   6   7   8   9  10  11  12  13  14  16  17  19 
## 829 203 158 119 116  84  68  55  55  19  13  14  13   4   4   2   3   1 
##  20  21  22 
##   1   1   2
\end{verbatim}

\begin{Shaded}
\begin{Highlighting}[]
\KeywordTok{table}\NormalTok{(}\KeywordTok{unlist}\NormalTok{(}\KeywordTok{sapply}\NormalTok{(unb.prof, }\ControlFlowTok{function}\NormalTok{(x) }\KeywordTok{length}\NormalTok{(x}\OperatorTok{$}\NormalTok{orientacoes_academicas}\OperatorTok{$}\NormalTok{ORIENTACAO_CONCLUIDA_POS_DOUTORADO}\OperatorTok{$}\NormalTok{ano))))}
\end{Highlighting}
\end{Shaded}

\begin{verbatim}
## 
##    0    1    2    3    4    5    6    7    8    9   10   14   16   17   21 
## 1497  109   66   28   23   10   11    7    3    2    2    3    1    1    1
\end{verbatim}

\begin{Shaded}
\begin{Highlighting}[]
\CommentTok{# Número de orientações por ano}
\KeywordTok{table}\NormalTok{(}\KeywordTok{unlist}\NormalTok{(}\KeywordTok{sapply}\NormalTok{(unb.prof, }\ControlFlowTok{function}\NormalTok{(x) (x}\OperatorTok{$}\NormalTok{orientacoes_academicas}\OperatorTok{$}\NormalTok{ORIENTACAO_CONCLUIDA_MESTRADO}\OperatorTok{$}\NormalTok{ano))))}
\end{Highlighting}
\end{Shaded}

\begin{verbatim}
## 
## 2010 2011 2012 2013 2014 2015 2016 2017 
##  962 1167 1288 1541 1621 1513 1502 1281
\end{verbatim}

\begin{Shaded}
\begin{Highlighting}[]
\KeywordTok{table}\NormalTok{(}\KeywordTok{unlist}\NormalTok{(}\KeywordTok{sapply}\NormalTok{(unb.prof, }\ControlFlowTok{function}\NormalTok{(x) (x}\OperatorTok{$}\NormalTok{orientacoes_academicas}\OperatorTok{$}\NormalTok{ORIENTACAO_CONCLUIDA_DOUTORADO}\OperatorTok{$}\NormalTok{ano))))}
\end{Highlighting}
\end{Shaded}

\begin{verbatim}
## 
## 2010 2011 2012 2013 2014 2015 2016 2017 
##  304  360  433  557  550  554  617  524
\end{verbatim}

\begin{Shaded}
\begin{Highlighting}[]
\KeywordTok{table}\NormalTok{(}\KeywordTok{unlist}\NormalTok{(}\KeywordTok{sapply}\NormalTok{(unb.prof, }\ControlFlowTok{function}\NormalTok{(x) (x}\OperatorTok{$}\NormalTok{orientacoes_academicas}\OperatorTok{$}\NormalTok{ORIENTACAO_CONCLUIDA_POS_DOUTORADO}\OperatorTok{$}\NormalTok{ano))))}
\end{Highlighting}
\end{Shaded}

\begin{verbatim}
## 
## 2010 2011 2012 2013 2014 2015 2016 2017 
##   75   66   95  103  134  106   98   63
\end{verbatim}

\subsubsection{Arquivo Publicação}\label{arquivo-publicacao}

\begin{Shaded}
\begin{Highlighting}[]
\CommentTok{# Visualizar a estrutura do arquivo de Publicacao}
\CommentTok{#jsonedit(unb.pub)}
\CommentTok{#Criando um data-frame com todos os anos}
\NormalTok{unb.pub.df <-}\StringTok{ }\KeywordTok{data.frame}\NormalTok{()}
\ControlFlowTok{for}\NormalTok{ (i }\ControlFlowTok{in} \DecValTok{1}\OperatorTok{:}\KeywordTok{length}\NormalTok{(unb.pub[[}\DecValTok{1}\NormalTok{]]))}
\NormalTok{  unb.pub.df <-}\StringTok{ }\KeywordTok{rbind}\NormalTok{(unb.pub.df, unb.pub}\OperatorTok{$}\NormalTok{PERIODICO[[i]])}
\KeywordTok{glimpse}\NormalTok{(unb.pub.df)}
\end{Highlighting}
\end{Shaded}

\begin{verbatim}
## Observations: 24,456
## Variables: 10
## $ natureza           <chr> "COMPLETO", "COMPLETO", "COMPLETO", "COMPLE...
## $ titulo             <chr> "An unusual presentation of pediatric Cushi...
## $ periodico          <chr> "Journal of Pediatric Endocrinology & Metab...
## $ ano                <chr> "2010", "2010", "2010", "2010", "2010", "20...
## $ volume             <chr> "23", "5", "78", "32", "13", "7", "259", "2...
## $ issn               <chr> "0334018X", "17446651", "00099163", "180611...
## $ paginas            <chr> "607 - 612", "697 - 709", "457 - 463", "1 -...
## $ doi                <chr> "", "10.1586/eem.10.47", "10.1111/j.1399-00...
## $ autores            <list> [<"AZEVEDO, M. F.;Azevedo, M;AZEVEDO, M F;...
## $ `autores-endogeno` <list> ["0017467628165816", "0017467628165816", "...
\end{verbatim}

\begin{Shaded}
\begin{Highlighting}[]
\CommentTok{# Limpando o data-frame de listas}
\NormalTok{unb.pub.df}\OperatorTok{$}\NormalTok{autores <-}\StringTok{ }\KeywordTok{gsub}\NormalTok{(}\StringTok{"}\CharTok{\textbackslash{}"}\StringTok{,}\CharTok{\textbackslash{}"}\StringTok{|}\CharTok{\textbackslash{}"}\StringTok{, }\CharTok{\textbackslash{}"}\StringTok{"}\NormalTok{, }\StringTok{"; "}\NormalTok{, unb.pub.df}\OperatorTok{$}\NormalTok{autores)}
\NormalTok{unb.pub.df}\OperatorTok{$}\NormalTok{autores <-}\StringTok{ }\KeywordTok{gsub}\NormalTok{(}\StringTok{"}\CharTok{\textbackslash{}"}\StringTok{|c}\CharTok{\textbackslash{}\textbackslash{}}\StringTok{(|}\CharTok{\textbackslash{}\textbackslash{}}\StringTok{)"}\NormalTok{, }\StringTok{""}\NormalTok{, unb.pub.df}\OperatorTok{$}\NormalTok{autores)}
\NormalTok{unb.pub.df}\OperatorTok{$}\StringTok{`}\DataTypeTok{autores-endogeno}\StringTok{`}\NormalTok{ <-}\StringTok{ }\KeywordTok{gsub}\NormalTok{(}\StringTok{","}\NormalTok{, }\StringTok{";"}\NormalTok{, unb.pub.df}\OperatorTok{$}\StringTok{`}\DataTypeTok{autores-endogeno}\StringTok{`}\NormalTok{)}
\NormalTok{unb.pub.df}\OperatorTok{$}\StringTok{`}\DataTypeTok{autores-endogeno}\StringTok{`}\NormalTok{ <-}\StringTok{ }\KeywordTok{gsub}\NormalTok{(}\StringTok{"}\CharTok{\textbackslash{}"}\StringTok{|c}\CharTok{\textbackslash{}\textbackslash{}}\StringTok{(|}\CharTok{\textbackslash{}\textbackslash{}}\StringTok{)"}\NormalTok{, }\StringTok{""}\NormalTok{, unb.pub.df}\OperatorTok{$}\StringTok{`}\DataTypeTok{autores-endogeno}\StringTok{`}\NormalTok{)}
\KeywordTok{glimpse}\NormalTok{(unb.pub.df)}
\end{Highlighting}
\end{Shaded}

\begin{verbatim}
## Observations: 24,456
## Variables: 10
## $ natureza           <chr> "COMPLETO", "COMPLETO", "COMPLETO", "COMPLE...
## $ titulo             <chr> "An unusual presentation of pediatric Cushi...
## $ periodico          <chr> "Journal of Pediatric Endocrinology & Metab...
## $ ano                <chr> "2010", "2010", "2010", "2010", "2010", "20...
## $ volume             <chr> "23", "5", "78", "32", "13", "7", "259", "2...
## $ issn               <chr> "0334018X", "17446651", "00099163", "180611...
## $ paginas            <chr> "607 - 612", "697 - 709", "457 - 463", "1 -...
## $ doi                <chr> "", "10.1586/eem.10.47", "10.1111/j.1399-00...
## $ autores            <chr> "AZEVEDO, M. F.;Azevedo, M;AZEVEDO, M F;AZE...
## $ `autores-endogeno` <chr> "0017467628165816", "0017467628165816", "00...
\end{verbatim}

\subsubsection{Arquivo Orientação}\label{arquivo-orientacao}

\begin{Shaded}
\begin{Highlighting}[]
\CommentTok{#Orientação}
\CommentTok{#Visualizar a estrutura do json no painel Viewer}
\CommentTok{#jsonedit(unb.adv)}
\CommentTok{#Reunir todos os anos e orientações concluidas em um mesmo data-frame}
\NormalTok{unb.adv.tipo.df <-}\StringTok{ }\KeywordTok{data.frame}\NormalTok{(); unb.adv.df <-}\StringTok{ }\KeywordTok{data.frame}\NormalTok{()}
\ControlFlowTok{for}\NormalTok{ (i }\ControlFlowTok{in} \DecValTok{1}\OperatorTok{:}\KeywordTok{length}\NormalTok{(unb.adv[[}\DecValTok{1}\NormalTok{]]))}
\NormalTok{  unb.adv.tipo.df <-}\StringTok{ }\KeywordTok{rbind}\NormalTok{(unb.adv.tipo.df, unb.adv}\OperatorTok{$}\NormalTok{ORIENTACAO_CONCLUIDA_POS_DOUTORADO[[i]])}
\NormalTok{unb.adv.df <-}\StringTok{ }\KeywordTok{rbind}\NormalTok{(unb.adv.df, unb.adv.tipo.df); unb.adv.tipo.df <-}\StringTok{ }\KeywordTok{data.frame}\NormalTok{()}
\ControlFlowTok{for}\NormalTok{ (i }\ControlFlowTok{in} \DecValTok{1}\OperatorTok{:}\KeywordTok{length}\NormalTok{(unb.adv[[}\DecValTok{1}\NormalTok{]]))}
\NormalTok{  unb.adv.tipo.df <-}\StringTok{ }\KeywordTok{rbind}\NormalTok{(unb.adv.tipo.df, unb.adv}\OperatorTok{$}\NormalTok{ORIENTACAO_CONCLUIDA_DOUTORADO[[i]])}
\NormalTok{unb.adv.df <-}\StringTok{ }\KeywordTok{rbind}\NormalTok{(unb.adv.df, unb.adv.tipo.df); unb.adv.tipo.df <-}\StringTok{ }\KeywordTok{data.frame}\NormalTok{()}
\ControlFlowTok{for}\NormalTok{ (i }\ControlFlowTok{in} \DecValTok{1}\OperatorTok{:}\KeywordTok{length}\NormalTok{(unb.adv[[}\DecValTok{1}\NormalTok{]]))}
\NormalTok{  unb.adv.tipo.df <-}\StringTok{ }\KeywordTok{rbind}\NormalTok{(unb.adv.tipo.df, unb.adv}\OperatorTok{$}\NormalTok{ORIENTACAO_CONCLUIDA_MESTRADO[[i]])}
\NormalTok{unb.adv.df <-}\StringTok{ }\KeywordTok{rbind}\NormalTok{(unb.adv.df, unb.adv.tipo.df)}
\KeywordTok{glimpse}\NormalTok{(unb.adv.df)}
\end{Highlighting}
\end{Shaded}

\begin{verbatim}
## Observations: 15,109
## Variables: 13
## $ natureza                    <chr> "Supervisão de pós-doutorado", "Su...
## $ titulo                      <chr> "A Interculturalidade na sala de a...
## $ ano                         <chr> "2010", "2010", "2010", "2010", "2...
## $ id_lattes_aluno             <chr> "", "", "", "", "", "", "", "", ""...
## $ nome_aluno                  <chr> "Lucielena Mendonça de Lima", "Iov...
## $ instituicao                 <chr> "Universidad de Brasília", "Univer...
## $ curso                       <chr> "", "", "", "", "", "", "", "", ""...
## $ codigo_do_curso             <chr> "", "", "", "", "", "", "", "", ""...
## $ bolsa                       <chr> "NAO", "SIM", "SIM", "SIM", "SIM",...
## $ agencia_financiadora        <chr> "", "Fundação de Ciência e Tecnolo...
## $ codigo_agencia_financiadora <chr> "", "005100000992", "000700000992"...
## $ nome_orientadores           <list> ["Maria Luisa Ortíz Alvarez", "Ma...
## $ id_lattes_orientadores      <list> ["0562632464695581", "05626324646...
\end{verbatim}

\begin{Shaded}
\begin{Highlighting}[]
\CommentTok{#Transformar as colunas de listas em caracteres eliminando c("")}
\NormalTok{unb.adv.df}\OperatorTok{$}\NormalTok{nome_orientadores <-}\StringTok{ }\KeywordTok{gsub}\NormalTok{(}\StringTok{"}\CharTok{\textbackslash{}"}\StringTok{|c}\CharTok{\textbackslash{}\textbackslash{}}\StringTok{(|}\CharTok{\textbackslash{}\textbackslash{}}\StringTok{)"}\NormalTok{, }\StringTok{""}\NormalTok{, unb.adv.df}\OperatorTok{$}\NormalTok{nome_orientadores)}
\NormalTok{unb.adv.df}\OperatorTok{$}\NormalTok{id_lattes_orientadores <-}\StringTok{ }\KeywordTok{gsub}\NormalTok{(}\StringTok{"}\CharTok{\textbackslash{}"}\StringTok{|c}\CharTok{\textbackslash{}\textbackslash{}}\StringTok{(|}\CharTok{\textbackslash{}\textbackslash{}}\StringTok{)"}\NormalTok{, }\StringTok{""}\NormalTok{, unb.adv.df}\OperatorTok{$}\NormalTok{id_lattes_orientadores)}
\CommentTok{#Separar as colunas com dois orientadores}
\NormalTok{unb.adv.df <-}\StringTok{ }\KeywordTok{separate}\NormalTok{(unb.adv.df, nome_orientadores, }\DataTypeTok{into =} \KeywordTok{c}\NormalTok{(}\StringTok{"ori1"}\NormalTok{, }\StringTok{"ori2"}\NormalTok{), }\DataTypeTok{sep =} \StringTok{","}\NormalTok{)}
\end{Highlighting}
\end{Shaded}

\begin{verbatim}
## Warning: Expected 2 pieces. Additional pieces discarded in 6 rows [34, 35,
## 36, 2771, 5282, 5283].
\end{verbatim}

\begin{verbatim}
## Warning: Expected 2 pieces. Missing pieces filled with `NA` in 14710
## rows [1, 2, 3, 4, 5, 6, 7, 8, 9, 10, 11, 12, 13, 14, 15, 16, 17, 18, 19,
## 21, ...].
\end{verbatim}

\begin{Shaded}
\begin{Highlighting}[]
\NormalTok{unb.adv.df <-}\StringTok{ }\KeywordTok{separate}\NormalTok{(unb.adv.df, id_lattes_orientadores, }\DataTypeTok{into =} \KeywordTok{c}\NormalTok{(}\StringTok{"idLattes1"}\NormalTok{, }\StringTok{"idLattes2"}\NormalTok{), }\DataTypeTok{sep =} \StringTok{","}\NormalTok{)}
\end{Highlighting}
\end{Shaded}

\begin{verbatim}
## Warning: Expected 2 pieces. Additional pieces discarded in 6 rows [34, 35,
## 36, 2771, 5282, 5283].

## Warning: Expected 2 pieces. Missing pieces filled with `NA` in 14710
## rows [1, 2, 3, 4, 5, 6, 7, 8, 9, 10, 11, 12, 13, 14, 15, 16, 17, 18, 19,
## 21, ...].
\end{verbatim}

\begin{Shaded}
\begin{Highlighting}[]
\CommentTok{#Numero de orientacoes por ano}
\KeywordTok{table}\NormalTok{(unb.adv.df}\OperatorTok{$}\NormalTok{ano)}
\end{Highlighting}
\end{Shaded}

\begin{verbatim}
## 
## 2010 2011 2012 2013 2014 2015 2016 2017 
## 1301 1556 1749 2150 2237 2117 2166 1833
\end{verbatim}

\begin{Shaded}
\begin{Highlighting}[]
\CommentTok{#Tabela com nome de professor e numero de orientacoes}
\KeywordTok{head}\NormalTok{(}\KeywordTok{sort}\NormalTok{(}\KeywordTok{table}\NormalTok{(}\KeywordTok{rbind}\NormalTok{(unb.adv.df}\OperatorTok{$}\NormalTok{ori1, unb.adv.df}\OperatorTok{$}\NormalTok{ori2)), }\DataTypeTok{decreasing =} \OtherTok{TRUE}\NormalTok{), }\DecValTok{20}\NormalTok{)}
\end{Highlighting}
\end{Shaded}

\begin{verbatim}
## 
##                        Octavio Luiz Franco 
##                                         76 
##              Célia Maria de Almeida Soares 
##                                         61 
##                  Maria Fatima Grossi de Sa 
##                                         59 
##                     Jorge Madeira Nogueira 
##                                         53 
##         Concepta Margaret McManus Pimentel 
##                                         51 
##                    Juliana de Fátima Sales 
##                                         44 
##                Ana Maria Resende Junqueira 
##                                         40 
##                               Debora Diniz 
##                                         39 
##            Ana Suelly Arruda Câmara Cabral 
##                                         36 
##                          Gabriele Cornelli 
##                                         36 
##                         Helena Eri Shimizu 
##                                         36 
##                         Nivaldo dos Santos 
##                                         36 
##                         Lucio França Teles 
##                                         35 
##                      Edson Silva de Farias 
##                                         34 
##                      Ileno Izídio da Costa 
##                                         34 
##                  Ricardo Bentes de Azevedo 
##                                         34 
## Stella Maris Bortoni de Figueiredo Ricardo 
##                                         34 
##                  Anderson de Rezende Rocha 
##                                         33 
##                   Aparecido Divino da Cruz 
##                                         33 
##                     André Pacheco de Assis 
##                                         32
\end{verbatim}

\subsection{CRISP-DM Fase.Atividade 2.4 - Verificação da qualidade dos
dados.}\label{crisp-dm-fase.atividade-2.4---verificacao-da-qualidade-dos-dados.}

Como já informado, a verificação da qualidade dos dados envolve
responder se os dados disponíveis estão realmente completos.

As informações disponíveis são suficientes para o trabalho proposto?

Neste projeto, a verificação da qualidade dos dados é responsabilidade
dos alunos.

\section{\texorpdfstring{CRISP-DM Fase 3 - \textbf{Preparação dos
Dados}}{CRISP-DM Fase 3 - Preparação dos Dados}}\label{crisp-dm-fase-3---preparacao-dos-dados}

Como já informado, na fase de \textbf{Preparação dos Dados} os
\emph{datasets} que serão utilizados em todo o trabalho são construídos
a partir dos dados brutos. Aqui os dados são ``filtrados'' retirando-se
partes que não interessam e selecionando-se os ``campos'' necessários
para o trabalho de mineração.

São 5 as atividades genéricas nesta fase de preparação dos dados, a
seguir divididas em subseções

\subsection{CRISP-DM Fase.Atividade 3.1 - Seleção dos
dados.}\label{crisp-dm-fase.atividade-3.1---selecao-dos-dados.}

Como já informado, a seleção dos dados envolve identificar quais dados,
da nossa ``montanha de dados'', serão realmente utilizados.

Quais variáveis dos dados brutos serão convertidas para o
\emph{dataset}?

Não é raro cometer o erro de selecionar dados para um modelo preditivo
com base em uma falsa ideia de que aqueles dados contém a resposta para
o modelo que se quer construir. Surge o cuidado de se separar o sinal do
ruído (Silver, Nate. The Signal and the Noise: Why so many predictions
fail --- but some don't. USA: The Penguin Press HC, 2012.).

\subsection{CRISP-DM Fase.Atividade 3.2 - Limpeza dos
dados}\label{crisp-dm-fase.atividade-3.2---limpeza-dos-dados}

\subsection{CRISP-DM Fase.Atividade 3.3 - Construção dos
dados}\label{crisp-dm-fase.atividade-3.3---construcao-dos-dados}

Como já informado, a construção dos dados envolve a criação de novas
variáveis a partir de outras presentes nos \emph{datasets}.

\begin{Shaded}
\begin{Highlighting}[]
\CommentTok{# Funcoes }

\CommentTok{# converte as colunas de um dataframe tipo lista em tipo character}
\NormalTok{cv_tplista2tpchar <-}\StringTok{ }\ControlFlowTok{function}\NormalTok{( df  ) \{ }
  \ControlFlowTok{for}\NormalTok{( variavel }\ControlFlowTok{in} \KeywordTok{names}\NormalTok{(df)) \{}
    \ControlFlowTok{if}\NormalTok{ (}\KeywordTok{class}\NormalTok{(df[[variavel]]) }\OperatorTok{==}\StringTok{ "list"}\NormalTok{ ) \{}
\NormalTok{      df[[variavel]] <-}\StringTok{ }\KeywordTok{lapply}\NormalTok{(df[[variavel]] ,   }\ControlFlowTok{function}\NormalTok{(x)   }\KeywordTok{lista2texto}\NormalTok{( x  ) ) }
\NormalTok{      df[[variavel]] <-}\StringTok{ }\KeywordTok{as.character}\NormalTok{( df[[variavel]] )}
\NormalTok{    \}}
\NormalTok{  \}}
  \KeywordTok{return}\NormalTok{(df)}
\NormalTok{\}}
\NormalTok{###}


\CommentTok{# converte o conteudo de lista em array de characters}
\NormalTok{lista2texto <-}\StringTok{ }\ControlFlowTok{function}\NormalTok{( lista  ) \{}
  \ControlFlowTok{if}\NormalTok{(}\KeywordTok{is.null}\NormalTok{(lista)) \{}
    \KeywordTok{return}\NormalTok{ ( }\OtherTok{NULL}\NormalTok{ )}
\NormalTok{  \}}
\NormalTok{  saida <-}\StringTok{ ""}
  \ControlFlowTok{for}\NormalTok{( j }\ControlFlowTok{in} \DecValTok{1}\OperatorTok{:}\KeywordTok{length}\NormalTok{(lista)) \{ }
    \ControlFlowTok{for}\NormalTok{( i }\ControlFlowTok{in} \DecValTok{1}\OperatorTok{:}\KeywordTok{length}\NormalTok{(lista[[j]]) ) \{}
\NormalTok{      elemento <-}\StringTok{ }\NormalTok{lista[[j]][i] }
      \ControlFlowTok{if}\NormalTok{( }\OperatorTok{!}\KeywordTok{is.null}\NormalTok{(elemento)) \{ }
        \ControlFlowTok{if}\NormalTok{( i }\OperatorTok{==}\StringTok{ }\KeywordTok{length}\NormalTok{(lista[[j]]) }\OperatorTok{&}\StringTok{ }\NormalTok{j }\OperatorTok{==}\StringTok{ }\KeywordTok{length}\NormalTok{(lista)  ) \{ }
          \CommentTok{# se for o ultimo elemento nao coloque o ponto e virgula no final            }
\NormalTok{          saida <-}\StringTok{ }\KeywordTok{paste0}\NormalTok{( saida , elemento  )}
\NormalTok{        \} }\ControlFlowTok{else}\NormalTok{ \{}
          \CommentTok{# enquanto nao for o ultimo coloque ; separando os elementos concatenados }
\NormalTok{          saida <-}\StringTok{ }\KeywordTok{paste0}\NormalTok{( saida , elemento , }\DataTypeTok{sep =} \StringTok{" ; "}\NormalTok{)}
\NormalTok{        \}}
\NormalTok{      \}  }
\NormalTok{    \}}
\NormalTok{  \}}
  \KeywordTok{return}\NormalTok{( saida )}
\NormalTok{\}}

\CommentTok{# Converte producao elattes separada por anos em um unico dataframe }
\NormalTok{converte_producao2dataframe<-}\StringTok{ }\ControlFlowTok{function}\NormalTok{( lista_producao ) \{}
\NormalTok{  df_saida <-}\StringTok{ }\OtherTok{NULL} 
  
  \ControlFlowTok{for}\NormalTok{( ano }\ControlFlowTok{in} \KeywordTok{names}\NormalTok{(lista_producao)) \{}
\NormalTok{    df_saida <-}\StringTok{ }\KeywordTok{rbind}\NormalTok{(df_saida , lista_producao[[ano]])}
\NormalTok{  \}}
  
  \CommentTok{# converte tipo lista em array de character }
\NormalTok{  df_saida <-}\StringTok{ }\KeywordTok{cv_tplista2tpchar}\NormalTok{(df_saida)}
  \KeywordTok{return}\NormalTok{(df_saida)}
  

\NormalTok{\}}

\CommentTok{#concatena dois dataframes com  colunas diferentes }
\NormalTok{concatenadf <-}\StringTok{ }\ControlFlowTok{function}\NormalTok{( df1, df2) \{ }
  \CommentTok{#cria colunas de df1 que faltam em df2}
  \ControlFlowTok{for}\NormalTok{( coluna }\ControlFlowTok{in} \KeywordTok{names}\NormalTok{(df1 ) ) \{}
    \ControlFlowTok{if}\NormalTok{( }\OperatorTok{!}\KeywordTok{is.element}\NormalTok{(coluna, }\KeywordTok{names}\NormalTok{(df2) )) \{}
\NormalTok{      df2[coluna] <-}\StringTok{ }\OtherTok{NA}
\NormalTok{    \}}
\NormalTok{  \}}
  
  \CommentTok{#cria colunas de df2 que faltam em df1  }
  \ControlFlowTok{for}\NormalTok{( coluna }\ControlFlowTok{in} \KeywordTok{names}\NormalTok{(df2 ) ) \{}
    
    \ControlFlowTok{if}\NormalTok{( }\OperatorTok{!}\KeywordTok{is.element}\NormalTok{(coluna, }\KeywordTok{names}\NormalTok{(df1) )) \{}
\NormalTok{      df1[coluna] <-}\StringTok{ }\OtherTok{NA}
\NormalTok{    \}}
\NormalTok{  \}}
  
  
  \CommentTok{#faz o rbind dos dois dataframes }
\NormalTok{  df_final <-}\StringTok{ }\KeywordTok{rbind}\NormalTok{(df1 , df2)}
  \KeywordTok{return}\NormalTok{(df_final)}
  
\NormalTok{\}}

\CommentTok{# Extracao dos perfis dos professores }

\NormalTok{extrai_1perfil <-}\StringTok{ }\ControlFlowTok{function}\NormalTok{( professor ) \{}
\NormalTok{  idLattes <-}\StringTok{ }\KeywordTok{names}\NormalTok{(professor)}
\NormalTok{  nome <-}\StringTok{ }\NormalTok{professor[[idLattes]]}\OperatorTok{$}\NormalTok{nome   }
\NormalTok{  resumo_cv <-}\StringTok{ }\NormalTok{professor[[idLattes]]}\OperatorTok{$}\NormalTok{resumo_cv }
\NormalTok{  endereco_profissional <-}\StringTok{ }\NormalTok{professor[[idLattes]]}\OperatorTok{$}\NormalTok{endereco_profissional }\CommentTok{#list }
\NormalTok{  instituicao <-}\StringTok{ }\NormalTok{endereco_profissional}\OperatorTok{$}\NormalTok{instituicao}
\NormalTok{  orgao <-}\StringTok{ }\NormalTok{endereco_profissional}\OperatorTok{$}\NormalTok{orgao}
\NormalTok{  unidade <-}\StringTok{ }\NormalTok{endereco_profissional}\OperatorTok{$}\NormalTok{unidade}
\NormalTok{  DDD <-}\StringTok{ }\NormalTok{endereco_profissional}\OperatorTok{$}\NormalTok{DDD}
\NormalTok{  telefone <-}\StringTok{ }\NormalTok{endereco_profissional}\OperatorTok{$}\NormalTok{telefone}
\NormalTok{  bairro <-}\StringTok{ }\NormalTok{endereco_profissional}\OperatorTok{$}\NormalTok{bairro}
\NormalTok{  cep <-}\StringTok{ }\NormalTok{endereco_profissional}\OperatorTok{$}\NormalTok{cep}
\NormalTok{  cidade <-}\StringTok{ }\NormalTok{endereco_profissional}\OperatorTok{$}\NormalTok{cidade}
\NormalTok{  senioridade <-}\StringTok{ }\NormalTok{professor[[idLattes]]}\OperatorTok{$}\NormalTok{senioridade  }
\NormalTok{  df_1perfil <-}\StringTok{ }\KeywordTok{data.frame}\NormalTok{( idLattes , nome, resumo_cv ,instituicao , }
\NormalTok{                           orgao, unidade , DDD, telefone, bairro,cep,cidade , senioridade,}
                           \DataTypeTok{stringsAsFactors =} \OtherTok{FALSE}\NormalTok{)}
  
  \KeywordTok{return}\NormalTok{(df_1perfil)  }
\NormalTok{\}}

\NormalTok{extrai_perfis <-}\StringTok{ }\ControlFlowTok{function}\NormalTok{(jsonProfessores) \{}
\NormalTok{  df_saida <-}\StringTok{ }\KeywordTok{data.frame}\NormalTok{()}
  \ControlFlowTok{for}\NormalTok{( i }\ControlFlowTok{in} \DecValTok{1}\OperatorTok{:}\KeywordTok{length}\NormalTok{(jsonProfessores)) \{}
\NormalTok{    jsonProfessor <-}\StringTok{ }\NormalTok{jsonProfessores[i]}
\NormalTok{    df_professor <-}\StringTok{ }\KeywordTok{extrai_1perfil}\NormalTok{(jsonProfessor)}
    \ControlFlowTok{if}\NormalTok{( }\KeywordTok{nrow}\NormalTok{(df_saida) }\OperatorTok{>}\StringTok{ }\DecValTok{0}\NormalTok{ ) \{}
\NormalTok{      df_saida <-}\StringTok{ }\KeywordTok{rbind}\NormalTok{(df_saida , df_professor)}
\NormalTok{    \} }\ControlFlowTok{else}\NormalTok{ \{}
\NormalTok{      df_saida <-}\StringTok{ }\NormalTok{df_professor }
\NormalTok{    \}}
\NormalTok{  \}}
   
  \KeywordTok{return}\NormalTok{(df_saida)}
\NormalTok{\}}

\CommentTok{# Extracao da producao bibliografica dos professores }

\NormalTok{extrai_1producao <-}\StringTok{ }\ControlFlowTok{function}\NormalTok{(professor) \{}
\NormalTok{  idLattes <-}\StringTok{ }\KeywordTok{names}\NormalTok{(professor)}
\NormalTok{  df_1producao <<-}\StringTok{ }\OtherTok{NULL} 
\NormalTok{  producao_bibliografica <-}\StringTok{ }\NormalTok{professor[[idLattes]]}\OperatorTok{$}\NormalTok{producao_bibiografica  }\CommentTok{#list}
  \ControlFlowTok{for}\NormalTok{( tipo_producao }\ControlFlowTok{in} \KeywordTok{names}\NormalTok{(producao_bibliografica)) \{ }
\NormalTok{    df_temporario <-}\StringTok{ }\KeywordTok{cv_tplista2tpchar}\NormalTok{ ( producao_bibliografica[[tipo_producao]]) }
\NormalTok{    df_temporario}\OperatorTok{$}\NormalTok{tipo_producao <-}\StringTok{  }\NormalTok{tipo_producao }
\NormalTok{    df_temporario}\OperatorTok{$}\NormalTok{idLattes <-}\StringTok{  }\NormalTok{idLattes}
\NormalTok{    df_1producao <-}\StringTok{ }\KeywordTok{concatenadf}\NormalTok{( df_1producao , df_temporario  )}
\NormalTok{  \}  }
  \KeywordTok{return}\NormalTok{(df_1producao)}
\NormalTok{\}}

\NormalTok{extrai_producoes <-}\StringTok{ }\ControlFlowTok{function}\NormalTok{( jsonProfessores) \{}
\NormalTok{  df_saida <-}\StringTok{ }\KeywordTok{data.frame}\NormalTok{()}
  \ControlFlowTok{for}\NormalTok{( i }\ControlFlowTok{in} \DecValTok{1}\OperatorTok{:}\KeywordTok{length}\NormalTok{(jsonProfessores)) \{}
\NormalTok{    jsonProfessor <-}\StringTok{ }\NormalTok{jsonProfessores[i]}
\NormalTok{    df_producao <-}\StringTok{ }\KeywordTok{extrai_1producao}\NormalTok{(jsonProfessor)}
    \ControlFlowTok{if}\NormalTok{( }\KeywordTok{nrow}\NormalTok{(df_saida) }\OperatorTok{>}\StringTok{ }\DecValTok{0}\NormalTok{ ) \{}
\NormalTok{      df_saida <-}\StringTok{ }\KeywordTok{concatenadf}\NormalTok{(df_saida , df_producao)}
\NormalTok{    \} }\ControlFlowTok{else}\NormalTok{ \{}
\NormalTok{      df_saida <-}\StringTok{ }\NormalTok{df_producao }
\NormalTok{    \}}
\NormalTok{  \}}
\NormalTok{  df_saida <-}\StringTok{ }\NormalTok{df_saida }\OperatorTok\StringTok{ }\KeywordTok{filter}\NormalTok{( }\OperatorTok{!}\KeywordTok{is.na}\NormalTok{(tipo_producao))}
  \KeywordTok{return}\NormalTok{(df_saida)  }
\NormalTok{\}}

\CommentTok{# Extracao das orientacoes dos professores }

\NormalTok{extrai_1orientacao <-}\StringTok{ }\ControlFlowTok{function}\NormalTok{(professor) \{}
\NormalTok{  idLattes <-}\StringTok{ }\KeywordTok{names}\NormalTok{(professor)}
\NormalTok{  df_1orientacao <-}\StringTok{ }\OtherTok{NULL}
\NormalTok{  orientacoes_academicas  <-}\StringTok{ }\NormalTok{professor[[idLattes]]}\OperatorTok{$}\NormalTok{orientacoes_academicas  }\CommentTok{#list}
  \ControlFlowTok{for}\NormalTok{( orientacao }\ControlFlowTok{in} \KeywordTok{names}\NormalTok{(orientacoes_academicas )) \{ }
\NormalTok{    df_temporario <-}\StringTok{ }\KeywordTok{cv_tplista2tpchar}\NormalTok{ ( orientacoes_academicas[[orientacao]])}
\NormalTok{    df_temporario}\OperatorTok{$}\NormalTok{orientacao <-}\StringTok{  }\NormalTok{orientacao }
\NormalTok{    df_temporario}\OperatorTok{$}\NormalTok{idLattes <-}\StringTok{  }\NormalTok{idLattes}
\NormalTok{    df_1orientacao <-}\StringTok{ }\KeywordTok{concatenadf}\NormalTok{( df_1orientacao , df_temporario  )}
\NormalTok{  \}  }
  \KeywordTok{return}\NormalTok{(df_1orientacao) }
\NormalTok{\}}

\NormalTok{extrai_orientacoes <-}\StringTok{ }\ControlFlowTok{function}\NormalTok{(jsonProfessores) \{}
\NormalTok{  df_saida <-}\StringTok{ }\KeywordTok{data.frame}\NormalTok{()}
  \ControlFlowTok{for}\NormalTok{( i }\ControlFlowTok{in} \DecValTok{1}\OperatorTok{:}\KeywordTok{length}\NormalTok{(jsonProfessores)) \{}
\NormalTok{    jsonProfessor <-}\StringTok{ }\NormalTok{jsonProfessores[i]}
\NormalTok{    df_orientacao <-}\StringTok{ }\KeywordTok{extrai_1orientacao}\NormalTok{(jsonProfessor)}
    \ControlFlowTok{if}\NormalTok{( }\KeywordTok{nrow}\NormalTok{(df_saida) }\OperatorTok{>}\StringTok{ }\DecValTok{0}\NormalTok{ ) \{}
\NormalTok{      df_saida <-}\StringTok{ }\KeywordTok{concatenadf}\NormalTok{(df_saida , df_orientacao)}
\NormalTok{    \} }\ControlFlowTok{else}\NormalTok{ \{}
\NormalTok{      df_saida <-}\StringTok{ }\NormalTok{df_orientacao}
\NormalTok{    \}}
\NormalTok{  \}}
\NormalTok{  df_saida <-}\StringTok{ }\NormalTok{df_saida }\OperatorTok\StringTok{ }\KeywordTok{filter}\NormalTok{(}\OperatorTok{!}\KeywordTok{is.na}\NormalTok{(idLattes))}
  \KeywordTok{return}\NormalTok{(df_saida)  }
\NormalTok{\}}

\CommentTok{# Extracao das areas de atuacao dos professores }

\NormalTok{extrai_1area_de_atuacao <-}\StringTok{ }\ControlFlowTok{function}\NormalTok{(professor)\{}
\NormalTok{  idLattes <-}\StringTok{ }\KeywordTok{names}\NormalTok{(professor)}
\NormalTok{  df_1area <-}\StringTok{  }\NormalTok{professor[[idLattes]]}\OperatorTok{$}\NormalTok{areas_de_atuacao}
\NormalTok{  df_1area}\OperatorTok{$}\NormalTok{idLattes <-}\StringTok{ }\NormalTok{idLattes}
  \KeywordTok{return}\NormalTok{(df_1area)}
\NormalTok{\}}

\NormalTok{extrai_areas_atuacao <-}\StringTok{ }\ControlFlowTok{function}\NormalTok{(jsonProfessores)\{}
\NormalTok{  df_saida <-}\StringTok{ }\KeywordTok{data.frame}\NormalTok{()}
  \ControlFlowTok{for}\NormalTok{( i }\ControlFlowTok{in} \DecValTok{1}\OperatorTok{:}\KeywordTok{length}\NormalTok{(jsonProfessores)) \{}
\NormalTok{    jsonProfessor <-}\StringTok{ }\NormalTok{jsonProfessores[i]}
\NormalTok{    df_area_atuacao <-}\StringTok{ }\KeywordTok{extrai_1area_de_atuacao}\NormalTok{(jsonProfessor)}
    \ControlFlowTok{if}\NormalTok{( }\KeywordTok{nrow}\NormalTok{(df_saida) }\OperatorTok{>}\StringTok{ }\DecValTok{0}\NormalTok{ ) \{}
\NormalTok{      df_saida <-}\StringTok{ }\KeywordTok{concatenadf}\NormalTok{(df_saida , df_area_atuacao)}
\NormalTok{    \} }\ControlFlowTok{else}\NormalTok{ \{}
\NormalTok{      df_saida <-}\StringTok{ }\NormalTok{df_area_atuacao}
\NormalTok{    \}}
\NormalTok{  \}}
\NormalTok{  df_saida <-}\StringTok{ }\NormalTok{df_saida }\OperatorTok\StringTok{ }\KeywordTok{filter}\NormalTok{( }\OperatorTok{!}\KeywordTok{is.na}\NormalTok{(idLattes))}
  \KeywordTok{return}\NormalTok{(df_saida)   }
\NormalTok{\}}
\NormalTok{########################### Inicio }

\CommentTok{# colocar o diretorio onde está o arquivo json de perfis a serem lidos }
\NormalTok{unb.prof.json <-}\StringTok{ }\KeywordTok{read_file}\NormalTok{(}\StringTok{"data/unbpos.profile.json"}\NormalTok{)}
\NormalTok{unb.prof.df.capes <-}\StringTok{ }\KeywordTok{read.csv}\NormalTok{(}\StringTok{"data/PesqPosCapes.csv"}\NormalTok{, }
                              \DataTypeTok{sep =} \StringTok{";"}\NormalTok{, }\DataTypeTok{header =} \OtherTok{TRUE}\NormalTok{, }\DataTypeTok{colClasses =} \StringTok{"character"}\NormalTok{)}
\NormalTok{unb.prof <-}\StringTok{ }\KeywordTok{fromJSON}\NormalTok{(unb.prof.json)}
\KeywordTok{length}\NormalTok{(unb.prof)}
\end{Highlighting}
\end{Shaded}

\begin{verbatim}
## [1] 1764
\end{verbatim}

\begin{Shaded}
\begin{Highlighting}[]
\CommentTok{# extrai perfis dos professores }
\NormalTok{unb.prof.df.professores <-}\StringTok{ }\KeywordTok{extrai_perfis}\NormalTok{(unb.prof)}

\CommentTok{# extrai producao bibliografica de todos os professores }
\NormalTok{unb.prof.df.publicacoes <-}\StringTok{ }\KeywordTok{extrai_producoes}\NormalTok{(unb.prof)}

\CommentTok{#extrai orientacoes }
\NormalTok{unb.prof.df.orientacoes <-}\StringTok{ }\KeywordTok{extrai_orientacoes}\NormalTok{(unb.prof)}

\CommentTok{#extrai areas de atuacao }
\NormalTok{unb.prof.df.areas.de.atuacao <-}\StringTok{ }\KeywordTok{extrai_areas_atuacao}\NormalTok{(unb.prof)}

\CommentTok{#salva os daframes }
\KeywordTok{save}\NormalTok{(unb.prof.df.professores, unb.prof.df.publicacoes,}
\NormalTok{     unb.prof.df.orientacoes, unb.prof.df.areas.de.atuacao, }\DataTypeTok{file =} \StringTok{"dataframes.Rda"}\NormalTok{)}

\CommentTok{#cria arquivo para análise}
\NormalTok{unb.prof.df <-}\StringTok{ }\KeywordTok{data.frame}\NormalTok{()}
\NormalTok{unb.prof.df <-}\StringTok{ }\NormalTok{unb.prof.df.professores }\OperatorTok\StringTok{ }
\StringTok{  }\KeywordTok{select}\NormalTok{(idLattes, nome, resumo_cv, senioridade) }\OperatorTok\StringTok{ }
\StringTok{  }\KeywordTok{left_join}\NormalTok{(}
\NormalTok{    unb.prof.df.orientacoes }\OperatorTok\StringTok{ }
\StringTok{      }\KeywordTok{select}\NormalTok{(orientacao, idLattes) }\OperatorTok\StringTok{ }
\StringTok{      }\KeywordTok{filter}\NormalTok{(}\OperatorTok{!}\KeywordTok{grepl}\NormalTok{(}\StringTok{"EM_ANDAMENTO"}\NormalTok{, orientacao)) }\OperatorTok\StringTok{ }
\StringTok{      }\KeywordTok{group_by}\NormalTok{(idLattes) }\OperatorTok\StringTok{ }
\StringTok{      }\KeywordTok{count}\NormalTok{(orientacao) }\OperatorTok\StringTok{ }
\StringTok{      }\KeywordTok{spread}\NormalTok{(}\DataTypeTok{key =}\NormalTok{ orientacao, }\DataTypeTok{value =}\NormalTok{ n), }
    \DataTypeTok{by =} \StringTok{"idLattes"}\NormalTok{) }\OperatorTok\StringTok{ }
\StringTok{  }\KeywordTok{left_join}\NormalTok{(}
\NormalTok{    unb.prof.df.publicacoes }\OperatorTok\StringTok{ }
\StringTok{      }\KeywordTok{select}\NormalTok{(tipo_producao, idLattes) }\OperatorTok\StringTok{ }
\StringTok{      }\KeywordTok{filter}\NormalTok{(}\OperatorTok{!}\KeywordTok{grepl}\NormalTok{(}\StringTok{"ARTIGO_ACEITO"}\NormalTok{, tipo_producao)) }\OperatorTok\StringTok{ }
\StringTok{      }\KeywordTok{group_by}\NormalTok{(idLattes) }\OperatorTok\StringTok{ }
\StringTok{      }\KeywordTok{count}\NormalTok{(tipo_producao) }\OperatorTok\StringTok{ }
\StringTok{      }\KeywordTok{spread}\NormalTok{(}\DataTypeTok{key =}\NormalTok{ tipo_producao, }\DataTypeTok{value =}\NormalTok{ n), }
    \DataTypeTok{by =} \StringTok{"idLattes"}\NormalTok{) }\OperatorTok\StringTok{ }
\StringTok{  }\KeywordTok{left_join}\NormalTok{(}
\NormalTok{    unb.prof.df.areas.de.atuacao }\OperatorTok\StringTok{ }
\StringTok{      }\KeywordTok{select}\NormalTok{(area, idLattes) }\OperatorTok\StringTok{ }
\StringTok{      }\KeywordTok{group_by}\NormalTok{(idLattes) }\OperatorTok\StringTok{ }
\StringTok{      }\KeywordTok{summarise}\NormalTok{(}\KeywordTok{n_distinct}\NormalTok{(area)), }
    \DataTypeTok{by =} \StringTok{"idLattes"}\NormalTok{) }\OperatorTok\StringTok{ }
\StringTok{  }\KeywordTok{left_join}\NormalTok{(}
\NormalTok{    unb.prof.df.capes }\OperatorTok\StringTok{ }
\StringTok{      }\KeywordTok{select}\NormalTok{(AreaPos, idLattes) }\OperatorTok\StringTok{ }
\StringTok{      }\KeywordTok{group_by}\NormalTok{(idLattes) }\OperatorTok\StringTok{ }
\StringTok{      }\KeywordTok{summarise}\NormalTok{(}\KeywordTok{n_distinct}\NormalTok{(AreaPos)), }
    \DataTypeTok{by =} \StringTok{"idLattes"}\NormalTok{)}

\KeywordTok{glimpse}\NormalTok{(unb.prof.df)}
\end{Highlighting}
\end{Shaded}

\begin{verbatim}
## Observations: 1,764
## Variables: 16
## $ idLattes                               <chr> "0000507838194708", "00...
## $ nome                                   <chr> "Norai Romeu Rocco", "A...
## $ resumo_cv                              <chr> "Possui graduação em Ma...
## $ senioridade                            <chr> "8", "9", "7", "8", "9"...
## $ ORIENTACAO_CONCLUIDA_DOUTORADO         <int> 3, 3, NA, NA, NA, NA, N...
## $ ORIENTACAO_CONCLUIDA_MESTRADO          <int> 3, 14, 1, 5, 2, 3, NA, ...
## $ ORIENTACAO_CONCLUIDA_POS_DOUTORADO     <int> NA, NA, NA, NA, NA, NA,...
## $ OUTRAS_ORIENTACOES_CONCLUIDAS          <int> NA, 6, 11, 7, 5, 14, 10...
## $ CAPITULO_DE_LIVRO                      <int> NA, 3, 1, 5, 1, NA, 3, ...
## $ DEMAIS_TIPOS_DE_PRODUCAO_BIBLIOGRAFICA <int> 7, 10, NA, NA, NA, NA, ...
## $ EVENTO                                 <int> 1, 8, 25, 17, 9, 1, 26,...
## $ LIVRO                                  <int> NA, 2, NA, 2, NA, NA, 1...
## $ PERIODICO                              <int> 6, 27, 3, 6, 27, 2, 14,...
## $ TEXTO_EM_JORNAIS                       <int> NA, NA, NA, 1, NA, NA, ...
## $ `n_distinct(area)`                     <int> 2, 1, 2, 2, 1, 1, 1, 4,...
## $ `n_distinct(AreaPos)`                  <int> 1, 1, 1, 2, 1, 1, 1, 2,...
\end{verbatim}

\subsection{CRISP-DM Fase.Atividade 3.4 - Integração dos
dados}\label{crisp-dm-fase.atividade-3.4---integracao-dos-dados}

Como já informado, a integração dos dados envolve a união (merge) de
diferentes tabelas para criar um único \emph{dataset} para ser utilizado
no R, por exemplo.

\subsection{CRISP-DM Fase.Atividade 3.5 - Formatação dos
dados}\label{crisp-dm-fase.atividade-3.5---formatacao-dos-dados}

Como já informado, a formatação de dados envolve a realização de
pequenas alterações na estrutura dos dados, como a ordem das variáveis,
para permitir a execução de determinado método de data mining.

\section{\texorpdfstring{CRISP-DM Fase 4 -
\textbf{Modelagem}}{CRISP-DM Fase 4 - Modelagem}}\label{crisp-dm-fase-4---modelagem}

Como já informado, na fase de \textbf{Modelagem} no CRISP-DM ocorre a
construção e avaliação de modelos estatísticos ou computacionais,
podendo ser realizada em quatro atividades genéricas, a seguir
organizadas na forma de seções

\subsection{CRISP-DM Fase.Atividade 4.1 - Seleção das técnicas de
modelagem}\label{crisp-dm-fase.atividade-4.1---selecao-das-tecnicas-de-modelagem}

\subsection{CRISP-DM Fase.Atividade 4.2 - Realização de testes de
modelagem}\label{crisp-dm-fase.atividade-4.2---realizacao-de-testes-de-modelagem}

Como já informado, na realização de testes de modelagem diferentes
modelos estatísticos ou computacionais são previamente testados e
avaliados. Pode-se dividir o \emph{dataset} criado na etapa anterior
para se ter uma base de treino na construção de modelos, e outra pequena
parte para validar e avaliar a eficiência de cada modelo criado até se
chegar ao mais ``eficiente''.

\subsection{CRISP-DM Fase.Atividade 4.3 - Construção do modelo
definitivo}\label{crisp-dm-fase.atividade-4.3---construcao-do-modelo-definitivo}

Como já informado, a construçao do modelo definitivo é realizada com
base na melhor experiência do passo anterior.

\subsection{CRISP-DM Fase.Atividade 4.4 - Avaliação do
modelo}\label{crisp-dm-fase.atividade-4.4---avaliacao-do-modelo}

\section{\texorpdfstring{CRISP-DM Fase 5 -
\textbf{Avaliação}}{CRISP-DM Fase 5 - Avaliação}}\label{crisp-dm-fase-5---avaliacao}

Como já informado, na fase de \textbf{Avaliação} do CRISP-DM os
resultados não são apenas avaliados, mas se verifica se existem questões
relacionadas à organização que não foram suficientemente abordadas.
Deve-se refletir se o uso arepetido do modelo criado pode trazer algum
``efeito colateral'' para a organização.

Como já informado, nesta fase, pode-se trabalhar com 3 atividades
genéricas, a seguir distribuídas em seções.

\subsection{CRISP-DM Fase.Atividade 5.1 - Avaliação dos
resultados}\label{crisp-dm-fase.atividade-5.1---avaliacao-dos-resultados}

\subsection{CRISP-DM Fase.Atividade 5.2 - Revisão do
processo}\label{crisp-dm-fase.atividade-5.2---revisao-do-processo}

Como já informado, durante a revisão do processo verifica-se se o modelo
foi construído adequadamente. As variáveis (passadas) para construir o
modelo estarão disponíveis no futuro?

\subsection{CRISP-DM Fase.Atividade 5.3 - Determinação dos etapas
seguintes}\label{crisp-dm-fase.atividade-5.3---determinacao-dos-etapas-seguintes}

Como já informado, pode ser necessário decidir-se por finalizar o
projeto, passar à etapa de desenvolvimento, ou rever algumas fases
anteriores para a melhoria do projeto.

\section{\texorpdfstring{CRISP-DM Fase 6 - \textbf{Implantação}
(\emph{deployment})}{CRISP-DM Fase 6 - Implantação (deployment)}}\label{crisp-dm-fase-6---implantacao-deployment}

Como já informado, na fase de \textbf{Implantação} (\emph{deployment})
se realiza o planejamento de implantação dos produtos desenvolvidos
(scripts, no caso do executado nesta disciplina) para o ambiente
operacional, para seu uso repetitivo, envolvendo atividades de
monitoramento e manutenção do sistema (script) desenvolvido. A fase de
implantação concluir com a produção e apresentação do relatório final
com os resultados do projeto.

Como já informado, são seis as atividades genéricas na fase de
\textbf{implantação}, a seguir apresentadas na forma de seções.

\subsection{CRISP-DM Fase.Atividade 6.1 - Planejamento da
transição}\label{crisp-dm-fase.atividade-6.1---planejamento-da-transicao}

De que forma os produtos desenvolvidos pelo grupo poderiam ser colocados
em uso prático regular, na organização cliente?

\subsection{CRISP-DM Fase.Atividade 6.2 - Planejamento do monitoramento
dos
produtos}\label{crisp-dm-fase.atividade-6.2---planejamento-do-monitoramento-dos-produtos}

De que forma seria possível realizar o monitoramento do funcionamento
dos produtos em utilização no ambiente operacional?

\subsection{CRISP-DM Fase.Atividade 6.3 - Planejamento de
manuteção}\label{crisp-dm-fase.atividade-6.3---planejamento-de-manutecao}

que manutenções, ajustes, mudanças, poderia ter que ser eventualmente
realizadas no produto (scripts), quando em uso no ambiente operacional
do cliente?

\subsection{CRISP-DM Fase.Atividade 6.4 - Produção do relatório
final}\label{crisp-dm-fase.atividade-6.4---producao-do-relatorio-final}

A entrega do relatório do grupo, tomando como base este aqui, reflete a
execução desta etapa.

\subsection{CRISP-DM Fase.Atividade 6.5 - Apresentação do relatório
final}\label{crisp-dm-fase.atividade-6.5---apresentacao-do-relatorio-final}

Como já informado, não será feita apresentação do relatório, mas
esperamos que publicações científicas possam ser geradas com pelo seu
grupo, com o apoio dos professores da disciplina.

\subsection{CRISP-DM Fase.Atividade 6.6 - Revisão sobre a execução do
projeto}\label{crisp-dm-fase.atividade-6.6---revisao-sobre-a-execucao-do-projeto}

Deve-se fazer aqui o registro de lições aprendidas, bem como traçadas
perspectivas futuras de aprimoramento deste trabalho, da disciplina de
Ciência de Dados para Todos etc.

\section{Referências}\label{referencias}

\begin{itemize}
\tightlist
\item
  Azevedo, Mário Luiz Neves de, João Ferreira de Oliveira, e Afrânio
  Mendes Catani. ``O Sistema Nacional de Pós-Graduação (SNPG) e o Plano
  Nacional de Educação (PNE 2014-2024): regulação, avaliação e
  financiamento''. Revista Brasileira de Política e Administração da
  Educação 32, nº 3 (2016).
  \url{http://dx.doi.org/10.21573/vol32n32016.68576}.
\item
  Can, Fazli, Tansel Özyer, e Faruk Polat, orgs. State of the Art
  Applications of Social Network Analysis. Lecture Notes in Social
  Networks. Switzerland: Springer International Publishing, 2014.
\item
  CAPES. ``Documentos de Área''. CAPES.gov.br. Acessado 12 de junho de
  2018.
  \url{http://avaliacaoquadrienal.capes.gov.br/documentos-de-area}.
\item
  ---------. ``Plano Nacional de Pós-Graduação - PNPG 2011/2020 Vol.
  1''. Brasília - DF, dezembro de 2010.
  \url{http://www.capes.gov.br/images/stories/download/Livros-PNPG-Volume-I-Mont.pdf}.
\item
  ---------. ``Plano Nacional de Pós-Graduação - PNPG 2011/2020 Vol.
  2''. Brasília - DF, dezembro de 2010.
  \url{http://www.capes.gov.br/images/stories/download/PNPG_Miolo_V2.pdf}.
\item
  ---------. ``Sucupira: coleta de dados, docentes de pós-graduação
  stricto sensu no Brasil 2015''. CAPES - Banco de Metadados, 16 de
  março de 2016.
  \url{http://metadados.capes.gov.br/index.php/catalog/63}.
\item
  Chapman, Pete, Julian Clinton, Randy Kerber, Thomas Khabaza, Thomas
  Reinartz, Colin Shearer, e Rüdiger Wirth. ``CRISP-DM 1.0: Step-by-Step
  Data Mining Guide''. USA: CRISP-DM Consortium, 2000.
  \url{https://www.the-modeling-agency.com/crisp-dm.pdf}.
\item
  Datacamp. ``Machine Learning with R (Skill Track)''. Datacamp, 2018.
  \url{https://www.datacamp.com/tracks/machine-learning}.
\item
  Fernandes, Jorge H C, e Ricardo Barros Sampaio. ``DataScienceForAll''.
  Zotero, 13 de junho de 2018.
  \url{https://www.zotero.org/groups/2197167/datascienceforall}.
\item
  ---------. ``Especificação do Trabalho Final da Disciplina de Ciência
  de Dados para Todos 2017.2: Estudo sobre a visibilidade internacional
  da produção científica das pós-graduações vinculadas às áreas de
  conhecimento da CAPES, na Universidade de Brasília (Comunicação
  Interna)''. Disciplina 116297 - Tópicos Avançados em Computadores,
  turma D, do semestre 2017.2, do Departamento de Ciência da Computação
  do Instituto de Ciências Exatas da Universidade de Brasília, 28 de
  novembro de 2017.
  \url{https://aprender.ead.unb.br/pluginfile.php/474549/mod_resource/content/1/Estudo\%20da\%20Cie\%CC\%82ncia.pdf}.
\item
  Fernandes, Jorge H C, Ricardo Barros Sampaio, e João Ribas de Moura.
  ``Ciência de Dados para Todos (Data Science For All) - 2018.1 -
  Análise da Produção Científica e Acadêmica da Universidade de Brasília
  - Modelo de Relatório Final da Disciplina - Departamento de Ciência da
  Computação da UnB''. Disciplina 116297 - Tópicos Avançados em
  Computadores, turma D, do semestre 2018.1, do Departamento de Ciência
  da Computação do Instituto de Ciências Exatas da Universidade de
  Brasília, 13 de junho de 2018.
\item
  Frickel, Scott, e Kelly Moore. The New Political Sociology of Science:
  Institutions, Networks, and Power. Science and technology in society.
  USA: The University of Wisconsin Press, 2006.
\item
  Graduate Prospects Ltd. ``Job profile: Higher education lecturer'',
  2018.
  \url{https://www.prospects.ac.uk/job-profiles/higher-education-lecturer}.
\item
  Kalpazidou Schmidt, Evanthia, e Ebbe Krogh Graversen. ``Persistent
  factors facilitating excellence in research environments''. Higher
  Education 75, nº 2 (1º de fevereiro de 2018): 341--63.
  \url{https://doi.org/10.1007/s10734-017-0142-0}.
\item
  Kilduff, Martin, e Wenpin Tsai. Social Networks and Organizations. UK:
  Sage Publications, 2003.
\item
  Kolaczyk, Eric D., e Gábor Csárdi. Statistical Analysis of Network
  Data with R. USA: Springer, 2014.
\item
  Kuhn, Max, Jed Wing, Steve Weston, Andre Williams, Chris Keefer, Allan
  Engelhardt, Tony Cooper, et al. ``Package `Caret' - Classification and
  Regression Training'', 27 de maio de 2018.
  \url{https://cran.r-project.org/web/packages/caret/caret.pdf}.
\item
  Leite, Fernando César Lima. ``Considerações básicas sobre a Avaliação
  do Sistema Nacional de Pós-Graduação''. Comunicação Pessoal (slides).
  Universidade de Brasília, abril de 2018.
  \url{https://aprender.ead.unb.br/pluginfile.php/502250/mod_resource/content/1/Considera\%C3\%A7\%C3\%B5es\%20b\%C3\%A1sicas\%20sobre\%20a\%20Avalia\%C3\%A7\%C3\%A3o\%20do\%20Sistema\%20Nacional.pdf}.
\item
  Lusher, Dean, Johan Koskinen, e Garry Robins, orgs. Exponential Random
  Graph Models for Social Networks: Theory, methods, and applications.
  Structural Analysis in the Social Sciences. USA: Cambridge University
  Press, 2013.
\item
  Mariscal, Gonzalo, Óscar Marbán, e Covadonga Fernández. ``A survey of
  data mining and knowledge discovery process models and
  methodologies''. The Knowledge Engineering Review 25, nº 2 (2010):
  137--66. \url{https://doi.org/10.1017/S0269888910000032}.
\item
  Nery, Guilherme, Ana Paula Bragaglia, Flávia Clemente, e Suzana
  Barbosa. ``Nem tudo parece o que é: Entenda o que é plágio''.
  Instituto de Arte e Comunicação Social da UFF, 2009.
  \url{http://www.noticias.uff.br/arquivos/cartilha-sobre-plagio-academico.pdf}.
\item
  Nooy, Wouter de, Andrej Mrvar, e Vladimir Batagelj. Exploratory Social
  Network Analysis with Pajek. Structural Analysis in the Social
  Sciences. USA: Routldge, 2005.
\item
  Pátaro, Cristina Saitê de Oliveira, e Frank Antonio Mezzomo. ``Sistema
  Nacional de Pós-Graduação no Brasil: estrutura, resultados e desafios
  para política de Estado - Lívio Amaral''. Revista Educação e
  Linguagens 2, nº 3 (julho de 2013): 11--17.
\item
  Schwartzman, Simon. ``A Ciência da Ciência''. Ciência Hoje 2, nº 11
  (março de 1984): 54--59.
\item
  Silver, Nate. The Signal and the Noise: Why so many predictions fail
  --- but some don't. USA: The Penguin Press HC, 2012.
\item
  Vicari, Donatella, Akinori Okada, Giancarlo Ragozini, e Claus Wiehs.
  Analysis and Modeling of Complex Data in Behavioral and Social
  Sciences. Studies in Classifi cation, Data Analysis, and Knowledge
  Organization. Switzerland: Springer, 2014.
\item
  Wickham, Hadley, e Garrett Grolemund. R for Data Science: Import,
  Tidy, Transform, Visualize, and Model Data. USA: O'Reilly, 2016.
\end{itemize}


\end{document}
